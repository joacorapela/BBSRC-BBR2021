
List the main objectives of the proposal in order of priority [up to
  4000 chars]

The overall goal of this proposal is to extend, enhance, maintain and support Bonsai, a fully integrated software
environment to enable cutting-edge reproducible laboratory systems neuroscience experiments in animal models.

Specifically, we will:

1. Develop, test and deploy a new software workflow to incorporate external packages into Bonsai's C# environment.

2. Incorporate a suite of machine-learning-driven analysis tools into the Bonsai environment to implement:
2a. video-based  behavioural analysis
2b. real-time interfacing of population-scale activity with laboratory control  
2c. evaluate competing data analytic approaches within a single experimental framework

3. Enhance the quality and reproducibility of systems neuroscience through:
3a. fully open-source and public-facing development
3b. rapid release of, and community engagement around, new tools
3c. provision of standard implementations of analysis and control methods, with the ability to distribute automated
descriptions of analysis pipelines for reproducibility.

4. Extend and support the Bonsai user community through code and forum engagement and development workshops.

[2829 chars remaining]




Describe the Bioinformatic and/or Biological Resource in simple terms in a way that could be publicised to a general audience [up to 4000 chars]

Understanding the brain is a major scientific challenge of our era. To succeed, scientists must be able to explain how animal behaviour relates to neural activity across different brain regions. This requires careful design and manipulation of behavioural experiments, where experimenters either record or manipulate neural activity while the animals (e.g. non-human primates, rodents, fish, insects) perform particular tasks. 

Experiments in behavioural and brain science laboratories require software that integrates and controls hardware from multiple recording devices (video, electrodes for neural activity measurement, sensors), and analysis tools that can interpret large and complex behavioural and neural datasets. 

Scientists studying brain and behaviour dedicate the majority of their time designing experiments and analysing the data, with least amount spent on data acquisition itself, which may impact the quality of data. Moreover, hundreds of neuroscience research groups worldwide develop their own experimental and analytical tools, most using different programming languages, leading to  inefficiencies in data sharing and analysis, and impacting reproducibility (i.e. how easy it is for someone else to do the same experiment). 

Here we propose to provide the neuroscience community a software tool that will dramatically increase the efficiency of experimental control and data analysis. We will do so by developing a new set of functionalities to an existing software platform, Bonsai. Bonsai is a fully integrated software environment that emphasises performance, flexibility, and ease-of-use, allowing scientists with no previous programming experience to quickly develop their own high-performance data acquisition and experimental control systems.

Thus far, Bonsai has been adopted by hundreds of scientists worldwide to provide interactive experimental control in behavioural and brain sciences. In this proposal, we aim to extend Bonsai’s functionality with a toolbox of online and offline Machine Intelligence tools for analysis of behavioural and neural datasets, and to create an open-access platform for software sharing. Bonsai’s enhanced functionality will enable new types of research, and speed up discovery and improve efficiency by (i) providing access to such tools to laboratories lacking expertise, (ii) reducing the need to reinvent the same tools in multiple labs and (iii) standardising the data processing streams, thus increasing reproducibility across laboratories. We believe this effort will enable and accelerate new discoveries in how the brain generates behaviour. 



Describe the Bioinformatic and/or Biological Resource in a manner suitable for a specialist reader. This summary will be made publicly available if the proposal is funded. [up to 2000 characters]

To understand the brain, scientists aim to explain how animal behaviour relates to neural activity. This requires the design and precise control of behavioural experiments, wherein animals perform particular tasks while experimenters either record or manipulate neural activity in specific neural circuits. Such experiments require data acquisition software that integrates and controls hardware from multiple recording devices (cameras, electrodes, sensors), and analysis tools that can interpret large and complex datasets. Progress is held back by the lack of standardised tools for design and implementation of experimental protocols, and the difficulty of integrating state-of-the-art data processing and neuroinformatics into custom experimental designs. The fields of behavioural and brain sciences have consequently suffered from both inefficiency and poor reproducibility, due to disparate data acquisition and analysis solutions created independently across laboratories.   

To address these challenges, we propose to extend, enhance, maintain and support \textbf{Bonsai}, a fully integrated software environment to enable cutting-edge reproducible systems neuroscience experiments using animal models, with a particular emphasis on machine-intelligence-enabled, real-time neuroinformatics methods. While Bonsai is already adopted by hundreds of scientists worldwide, we aim to extend Bonsai’s functionality with a toolbox of online and offline Machine Intelligence tools for analysis of behavioural and neural data (video-based analysis of behavioural motifs, real-time and offline analysis of neural signals), and create an open-access platform for software sharing and integration with multiple programming languages. Enhancing Bonsai's ecosystem will be a game-changer for behavioural and brain science experiments by enabling new types of research, increasing and diversifying user base, and dramatically improve efficiency and reproducibility of research.


Describe who will benefit from this Bioinformatic and/or Biological resource(s) being funded [up to 4000 characters]

Resource is Bonsai.
Monitor animal welfare


