
List the main objectives of the proposal in order of priority [up to
  4000 chars]

The overall goal of this proposal is to extend, enhance, maintain and support Bonsai, a fully integrated software environment enabling cutting-edge reproducible neuroscience and behavioural science experiments, with machine-intelligence-enabled, real-time neuroinformatics methods.

Specifically, we will:

1. Develop, test and deploy new programming language interfaces to allow extending the Bonsai environment with new packages developed in languages other than C#, namely MATLAB, Python and R.

2. Define a standard API and set of operators appropriate for composing machine-learning functionality into a Bonsai workflow, specifically to enable:
2a. learning
2b. inference
2c. visualisation
2d. model-comparison and validation
2e. exchange of models and hyper-parameters

3. Incorporate a suite of machine-learning-driven analysis tools into the Bonsai environment using the standard API specified above, to implement:
3a. video-based behavioural analysis
3b. real-time interfacing of population-scale extracellular activity with laboratory control  
3c. evaluation of competing data analytic approaches within a single experimental framework

4. Enhance the quality and reproducibility of systems neuroscience through:
4a. fully open-source and public-facing development
4b. rapid release of, and community engagement around, new tools
4c. provision of standard implementations of analysis and control methods, with the ability to distribute fully automated and reproducible analysis pipelines.

5. Extend and support the Bonsai user community through code, documentation, and forum engagement and development workshops.

[2829 chars remaining]




Describe the Bioinformatic and/or Biological Resource in simple terms in a way that could be publicised to a general audience [up to 4000 chars]

Understanding the brain and the behaviour it generates is a major scientific challenge of our era. To succeed, scientists must be able to explain how animal behaviour relates to neural activity across different brain regions. This requires careful design and manipulation of behavioural experiments, where experimenters either record or manipulate neural activity while the animals (e.g. non-human primates, rodents, fish, insects) engage in specific behaviours which need to be carefully observed and quantified. 

Experiments in behavioural and brain science laboratories require software that integrates and controls hardware from multiple recording devices (video, electrodes for neural activity measurement, sensors), and analysis tools that can interpret large and complex behavioural and neural datasets. 

Scientists studying brain and behaviour dedicate the majority of their time designing experiments and analysing the data, with least time spent on data acquisition itself, which may impact the quality of data. Moreover, hundreds of neuroscience research groups worldwide develop their own experimental and analytical tools, most using different programming languages, leading to  inefficiencies in data sharing and analysis, and impacting reproducibility (i.e. how easy it is for someone else to repeat the same experiment). 

Here we propose to provide the scientific community with a software tool that will dramatically increase the efficiency of experimental control and data analysis. We will do so by developing a new set of functionalities to an existing software platform, Bonsai. Bonsai is a fully integrated software environment that emphasises performance, flexibility, and ease-of-use, allowing scientists with no previous programming experience to quickly develop their own high-performance data acquisition and experimental control systems.

Thus far, Bonsai has been adopted by hundreds of scientists worldwide to provide interactive experimental control in behavioural and brain sciences. In this proposal, we aim to extend Bonsai’s functionality with a toolbox of online and offline Machine Intelligence tools for analysis of behavioural and neural datasets, and to create an open-access platform for software sharing. Bonsai’s enhanced functionality will enable new types of research, and speed up discovery and improve efficiency by (i) providing access to such tools to laboratories lacking expertise, (ii) reducing the need to reinvent the same tools in multiple labs and (iii) standardising the data processing streams, thus increasing reproducibility across laboratories. We believe this effort will enable and accelerate new discoveries in how the brain generates behaviour. 



Describe the Bioinformatic and/or Biological Resource in a manner suitable for a specialist reader. This summary will be made publicly available if the proposal is funded. [up to 2000 characters]

To understand the brain, scientists aim to explain how animal behaviour relates to neural activity. This requires the design and precise control of behavioural experiments, wherein animals perform particular tasks while experimenters either record or manipulate neural activity in specific neural circuits. Such experiments require data acquisition software that integrates and controls hardware from multiple recording devices (cameras, electrodes, sensors), and analysis tools that can interpret large and complex datasets. Progress is held back by the lack of standardised tools for design and implementation of experimental protocols, and the difficulty of integrating state-of-the-art data processing and neuroinformatics into custom experimental designs. The fields of behavioural and brain sciences have consequently suffered from both inefficiency and poor reproducibility, due to disparate data acquisition and analysis solutions created independently across laboratories.   

To address these challenges, we propose to extend, enhance, maintain and support \textbf{Bonsai}, a fully integrated software environment to enable cutting-edge reproducible systems neuroscience experiments using animal models, with a particular emphasis on machine-intelligence-enabled, real-time neuroinformatics methods. While Bonsai is already adopted by hundreds of scientists worldwide, we aim to extend Bonsai’s functionality with a toolbox of online and offline Machine Intelligence tools for analysis of behavioural and neural data (video-based analysis of behavioural motifs, real-time and offline analysis of neural signals), and create an open-access platform for software sharing and integration with multiple programming languages. Enhancing Bonsai's ecosystem will be a game-changer for behavioural and brain science experiments by enabling new types of research, increasing and diversifying user base, and dramatically improve efficiency and reproducibility of research.


Describe who will benefit from this Bioinformatic and/or Biological resource(s) being funded [up to 4000 characters]

The main beneficiaries of the proposed resource will be experimental brain and animal behaviour labs worldwide who currently lack the highly technical expertise required to apply Machine Intelligence to their experiments. In fields like systems neuroscience, where progress is made by exploring the unknowns of animal behaviour and wide-scale brain physiology, the need for redesign of experimental protocols and rapid analysis of large and complex datasets is constant and critical. While increasingly-automated Machine Learning platforms are becoming available across many data science fields, they require deep expertise or training to construct and train new algorithms; or else they are easy to use but limited to very specific domains (e.g. face recognition, pose estimation). If successful, this proposal will empower an entirely new community of non-technical researchers to recombine and repurpose Machine Intelligence tools using a flexible visual programming environment and providing seamless integration with all the measurement, instrumentation and control packages already available in Bonsai.

Machine Learning and computational neuroscience experts will also benefit from a standard platform to compare and benchmark different algorithms and stress-test their tools against community datasets and experiments. The addition of new programming languages to develop and interface with Bonsai will also greatly expand the potential for application of their methods into new domains of research.

The scientific community will collectively benefit from gains in efficiency and reproducibility of deploying Machine Intelligence pipelines, and from allowing non-experts to compose existing algorithms in a flexible way to design entirely novel Machine Intelligence applications tailor-made for specific research questions. The proposed tools will generate cost savings to individual laboratories as well as funding agencies, by drastically reducing duplicated software development efforts across research groups.

In addition to the research community, the standard analysis pipelines developed in this project may be applied to other domains of society where tracking, monitoring and understanding animal behaviour would yield benefits and improved outcomes. For example, Bonsai is  increasingly used in research institutions worldwide to monitor animal welfare by in-house technicians who are concerned about a broader range of environmental variables which might impact animal husbandry but which cannot be monitored in standard vivarium solutions. As a free and open-source software package, Bonsai would empower such users to prototype and design new animal monitoring systems on-site, without the need to engage in expensive software development to answer simple questions. It could thus contribute to improving the welfare of hundreds of thousands of animals in research facilities, zoos or farms. More broadly, Bonsai may also be applied to quantify dynamics of objects in wider settings from drone or satellite imagery, to track and quantify the dynamics of crop growth, deforestation or other biological ecosystems.


