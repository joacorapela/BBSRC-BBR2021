
List the main objectives of the proposal in order of priority [up to
  4000 chars]

The overall goal of this proposal is to extend, enhance, maintain and support Bonsai, a fully integrated software
environment to enable cutting-edge reproducible laboratory systems neuroscience experiments in animal models.

Specifically, we will:

1. Develop, test and deploy a new software workflow to incorporate external packages into Bonsai's C# environment.

2. Incorporate a suite of machine-learning-driven analysis tools into the Bonsai environment to implement:
2a. video-based  behavioural analysis
2b. real-time interfacing of population-scale activity with laboratory control  
2c. evaluate competing data analytic approaches within a single experimental framework

3. Enhance the quality and reproducibility of systems neuroscience through:
3a. fully open-source and public-facing development
3b. rapid release of, and community engagement around, new tools
3c. provision of standard implementations of analysis and control methods, with the ability to distribute automated
descriptions of analysis pipelines for reproducibility.

4. Extend and support the Bonsai user community through code and forum engagement and development workshops.

[2829 chars remaining]




Describe the Bioinformatic and/or Biological Resource in simple terms in a way that could be publicised to a general audience [up to 4000 chars]

Understanding the brain is a major scientific challenge of our era. To succeed, scientists must be able to explain how the rich repertoire of animal behaviour relates to the activity in neural circuits across different brain regions. This necessitates careful design and control of behavioural experiments wherein experimental animals (e.g. non-human primates, rodents, fish, insects) perform particular tasks while experimenters either record or manipulate neural activity in candidate neural circuits. 
Experiments in behavioural and brain science laboratories require software that integrates and controls hardware from multiple recording devices (video, electrodes for neural activity measurement, sensors), and analysis tools that can interpret large and complex behavioural and neural datasets. 

Scientists studying brain and behaviour dedicate the majority of their time to the design of experimental software protocols that control hardware and to data analysis, with least amount spent on data acquisition itself. Moreover, hundreds of neuroscience research groups worldwide develop their own software tools for experimental control and data analysis in different programming languages, leading to inefficiencies resulting from time-loss spent on writing software (often by individuals lacking programming expertise), the re-invention of software with the same functionality and concomitant lack of experimental reproducibility across laboratories. 

Here we propose to provide a community software tool that will dramatically increase the efficiency of experimental control and data analysis by providing a new set of functionalities to an existing software platform, Bonsai. Bonsai is an open-source reactive visual programming environment that naturally orchestrates complex networks of sensors and hardware into workflows and allows integration of multiple programming languages into its framework. Thus far, Bonsai has been adopted by hundreds of scientists worldwide to provide interactive experimental control in behavioural and brain sciences. In this proposal, we aim to extend Bonsai’s functionality with a toolbox of online and offline Machine Intelligence tools for analysis of behavioural and neural datasets, and to create an open-access platform for software sharing. Bonsai’s enhanced functionality will enable new types of research via online experimental control, speed up discovery and improve efficiency by (i) providing access to such tools into laboratories lacking expertise, (ii) reducing the need to reinvent the same tools in multiple labs and (iii) standardising the data processing streams, thus increasing reproducibility across laboratories. We believe this effort will enable and accelerate new discoveries in how the brain generates behaviour. 



Describe the Bioinformatic and/or Biological Resource in a manner suitable for a specialist reader. This summary will be made publicly available if the proposal is funded. [up to 2000 characters]

To understand the brain, scientists aim to explain how the rich repertoire of animal behaviour relates to the activity in neural circuits across different brain regions. This necessitates careful design and precise control of behavioural experiments wherein experimental animals perform particular tasks while experimenters either record or manipulate neural activity in candidate neural circuits. Such experiments require software that integrates and controls hardware from multiple recording devices (cameras, electrodes, sensors), and analysis tools that can interpret large and complex behavioural and neural datasets. The lack of standardised tools for experimental control and analysis across the brain-and-behaviour community is incurring major inefficiencies, with disproportionate time allocated to software engineering and writing analysis code, and efforts often replicated across laboratories. Similarly, disparate experimental and analysis protocols makes is almost impossible to replicate experiments between research groups.
Bonsai is an open-source reactive visual programming environment that naturally orchestrates complex networks of sensors and hardware into workflows. Bonsai has been adopted by hundreds of scientists worldwide to provide interactive experimental control in behavioural and brain sciences. Here we aim to extend Bonsai’s functionality with a toolbox of online and offline Machine Intelligence tools for analysis of behavioural and neural data (video-based analysis of behavioural motifs, real-time and off-line analysis of neural signals), and create an open-access platform for software sharing and integration with multiple programming languages. Bonsai’s enhanced functionality will enable new types of research, accelerate discovery and improve efficiency by (i) providing access to such tools for scientists lacking expertise, (ii) reducing duplication of effort, and (iii) standardising the data processing streams, thus increasing reproducibility across laboratories.
[12 chars over]


Describe who will benefit from this Bioinformatic and/or Biological resource(s) being funded [up to 4000 characters]



