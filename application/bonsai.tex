
Bonsai~\citep{lopesEtAl15,lopesAndMonteiro21} is a free and open-source visual
programming language that emphasizes performance, flexibility, and ease-of-use,
allowing scientists with no previous programming experience to quickly develop
their own high-performance data acquisition and experimental control systems.
Bonsai combines a high-level event algebra for data streams with an integrated
development environment (IDE) and a strong library of plugins supporting
multiple hardware and software packages used by the biomedical research
community (Figures~\ref{fig:bonsai}a and~\ref{fig:bonsai}b).
%
A Bonsai graphical program consists of one or more source data streams
and several interconnected operators that transform input to output
datastreams (Figure~\ref{fig:bonsai}c).

Standard software tools for data analysis (ImageJ, MATLAB, R, etc.) have been
transformative to the progress of increasingly “data-rich” sciences. However,
the equivalent standardized software tools for data acquisition and control of
animal behaviour experiments are still lacking. Life science experiments demand
a combination of multiple instrumentation and control technologies, for both
behavioral and physiological investigations. The growing complexity on both the
amount of data that is collected, and the rich conditions under which behaviour
must be explored, place an increasing burden on experimenters to integrate
highly specialized equipment in unique configurations, while often lacking
expertise in the relevant engineering fields.

Bonsai has a large user base in the systems neuroscience community
(Figures~\ref{fig:bonsai}d-\ref{fig:bonsai}f).  In the last year alone, we estimate more than 1,000 new users have
started to incorporate Bonsai into their experimental protocols across the
world. The surprising rate of adoption of Bonsai in non-programmer experimental
labs highlights the need for accessible programming tools that enable
state-of-the-art technology but also allow researchers to stay in control and
change their experimental paradigms.  Many open-source software tools are
either inaccessible to non-programmers, or too constrained to be of general use
outside their narrow domain of application. Bonsai has been successful because
it has straddled this gap to some extent.

The language has also helped to potentiate the growing wave of foundational
open hardware initiatives, such as the OpenEphys \citep{siegleEtAl17} and UCLA
Miniscope~\citep{caiEtAl16}, making it possible to quickly combine and
integrate these tools into new experiments~\citep{buccinoEtAl18}.
%
Bonsai has been adopted in large neuroscience undertakings like the
International Brain
Laboratory\footnote{\href{https://www.internationalbrainlab.com/}{https://www.internationalbrainlab.com/}}
and the Allen Institute for Brain
Science\footnote{\href{https://alleninstitute.org/what-we-do/brain-science/}{https://alleninstitute.org/what-we-do/brain-science/}}.
%
More recently, Bonsai has started to expand outside the domain of neuroscience
into biomedical research and biotechnology tool development, and even outside
academia into public outreach and education programs.

\subsubsection*{Other resources in the subject area}

The space of technologies serving experimental control and behaviour monitoring
is large, and is traditionally occupied either by domain-specific graphical
user interfaces for control and recording of specific devices and experiment
types (e.g. Open Ephys
GUI\footnote{\href{https://open-ephys.org/gui/}{https://open-ephys.org/gui/}},
Miniscope DAQ
Software\footnote{\href{https://github.com/Aharoni-Lab/Miniscope-DAQ-QT-Software}{https://github.com/Aharoni-Lab/Miniscope-DAQ-QT-Software}})
or by real-time control frameworks for specifying task logic using state
machine or similar formalisms (e.g. NIMH ML
(nih.gov)\footnote{\href{https://monkeylogic.nimh.nih.gov/}{https://monkeylogic.nimh.nih.gov/}},
pyControl\footnote{\href{https://pycontrol.readthedocs.io/en/latest/}{https://pycontrol.readthedocs.io/en/latest/}},
Autopilot\footnote{\href{https://docs.auto-pi-lot.com/en/latest/}{https://docs.auto-pi-lot.com/en/latest/}},
Sanworks\footnote{\href{https://sanworks.io/index.php}{https://sanworks.io/index.php}}).
These dedicated interfaces are typically very comfortable for experimenters
operating within the specific domain that the tool is designed for, but tend to
become unwieldy when there is a need to introduce a new device or variation of
a task which is outside the scope of the framework. The alternative is to use a
more general programming language like Python or MATLAB, with the disadvantage
of the code being harder to understand, maintain, and change. Programming
languages like LabVIEW straddle the middle ground and provide a high-level and
flexible visual interface for composing data acquisition and control systems
themselves. Unlike Bonsai, however, the graphical elements of LabVIEW are
heterogeneous and very fine grained, creating the need for long and complicated
logical structures to implement even a simple experimental control system. By
providing an extremely simple, yet flexible, visual syntax, Bonsai opens the
opportunity even for complete non-programmers to design and successfully
customize relatively complex experiments from the ground up. It is mostly this
capability which has made Bonsai so attractive as a standard tool in
experimental neuroscience.


