\textbf{Bonsai}~\citep{lopesEtAl15,lopesAndMonteiro21} is a free and open-source visual programming language developed 
in response to these challenges.
% ~\citep{lopesEtAl15,lopesAndMonteiro21} is a free and open-source visual
% programming language that 
Its design emphasizes performance, flexibility, and ease-of-use,
allowing scientists with no previous programming experience to quickly develop
their own high-performance data acquisition and experimental control systems.
Bonsai combines a high-level event algebra for data streams with an integrated
development environment (IDE) and an extensive library of plugins supporting
multiple hardware and software packages used by the neuroscience research
community (Figure~\ref{fig:bonsai}A-B).
%
A Bonsai graphical program consists of one or more source data streams (e.g. neural activity, video, audio, sensors, etc)
and several interconnected operators that transform input to output
datastreams (Figure~\ref{fig:bonsai}C).

Standard software tools for data analysis (ImageJ, MATLAB, R, etc.) have been
transformative to the progress of increasingly “data-rich” sciences. However,
the equivalent standardized software tools for data acquisition and control of
animal behaviour neuroscience experiments are still lacking. Life science experiments demand
a combination of multiple instrumentation and control technologies, for both
behavioural and physiological investigations. The growing complexity on both the
amount of data that is collected, and the rich conditions under which behaviour
must be explored, place an increasing burden on experimenters to integrate
highly specialised equipment in unique configurations, while often lacking
expertise in the relevant engineering fields.

Reproducibility is a key element of Bonsai's design.  Data acquisition and experimental control protocols can be 
replicated in any laboratory by just sharing a Bonsai configuration file. 

Bonsai has been in adopted in hundreds of laboratories worldwide and has the largest user base in the systems neuroscience community (Figure~\ref{fig:bonsai}D-F).  In the last year alone, more than 1,000 new users incorporated Bonsai into their experimental protocols. The surprising rate of adoption of Bonsai in non-programmer experimental
labs highlights the need for accessible design tools that enable
state-of-the-art technology but also allow researchers to stay in control and flexibly
change their experimental paradigms.  Many open-source software tools are
either inaccessible to non-programmers, or too constrained to be of general use
outside their narrow domain of application. Bonsai has been successful because
it has straddled this gap.

The language has also helped to potentiate the growing wave of foundational
open hardware initiatives, such as the OpenEphys \citep{siegleEtAl17} and UCLA
Miniscope~\citep{caiEtAl16}, making it possible to quickly combine and
integrate these tools into new experiments~\citep{buccinoEtAl18}.
%
Bonsai has been adopted in large neuroscience undertakings like the
International Brain
Laboratory\footnote{\href{https://www.internationalbrainlab.com/}{https://www.internationalbrainlab.com/}}
and the Allen Institute for Brain
Science\footnote{\href{https://alleninstitute.org/what-we-do/brain-science/}{https://alleninstitute.org/what-we-do/brain-science/}}.
%
% LEAVE OUT FOR THIS GRANT:
% More recently, Bonsai has started to expand outside the domain of neuroscience
% into biomedical research and biotechnology tool development, and even outside
% academia into public outreach and education programs.


