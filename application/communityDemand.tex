% Bonsai has directly enabled high-impact research in neuroscience~\citep[e.g.,][]{soaresEtAl16,grundemannEtAl17,netoEtAl16,hagiharEtAla21,anguillonRodriguezEtAl21} and contributed to accelerating technical progress in the field by functioning as an integrator of open-source, community developed technologies \citep{itskovEtAl14,lopesEtAl21,kaneEtAl20}.
The lack of a strong and
flexible machine learning package capable of being easily integrated with
real-time experimental control workflows has been a recurrent concern in the
Bonsai user community. This has been brought up spontaneously multiple times in
the user community forums, both in the form of specific problems related to the
tracking and interpretation of online animal behaviour, and in the form of
explicit proposals and suggestions for the creation of a machine learning
package\footnote{https://groups.google.com/g/bonsai-users/c/BZ3zOOdv\_1c/m/x6OP75frBQAJ}. Even
relatively modest contributions such as integrating support for specific models
in Bonsai, such as DeepLabCut, or even the more specific zebrafish feature
tracking package BonZeb, has attracted tremendous attention and positive
feedback from the community \citep[e.g.,][]{kaneEtAl20,guilbeaultEtAl21}.

That there is a need for an easier interface into machine learning algorithms
is evidenced by the development of multiple graphical user interfaces for
tracking animal behaviour over the last several years, especially in the much
larger Python user community
\citep[e.g.,][]{walterAndCouzin21,guilbeaultEtAl21}. However, these graphical
user interfaces have no interaction with real-time experimental control, and
their support libraries require expertise with the Python programming language
to use flexibly. Bonsai provides a way of addressing this gap, using its
flexible visual programming language to combine ease-of-use and full
flexibility for experimental control.

We focused this proposal on neuroscience applications of Bonsai. However, the
addition of machine intelligence functionality is needed in many other
experimental areas where Bonsai is or could be used. Therefore, the software
abstractions that we will build to add to Bonsai machine intelligence
functionality for neuroscience experiments (e.g., methods to perform
multi-method comparisons) will translate to other application domains of
Bonsai.

The ongoing AI revolution makes it extremely timely to integrate these
technologies into Bonsai. The proposed software infrastructure will make very
easy the integration of novel machine learning methods into the Bonsai
ecosystem. This integration will be critical to unlock new understanding, value
and scientific leads from the rich and complex behavioral and neural
experimetal data that Bonsai makes possible to collect. In addition, this
integration will open to non-programmer experimentalists in the biological
sciences unprecedented opportunities to combine real-time machine learning with
experimental control in neuroscience, which has to the best of our knowledge
not really been done before in such a broad and comprehensive scale. We believe
addressing this bottleneck and enabling these technologies to non-programmers
will make this resource into a transformative and fundamental technology for
the next generation studies in animal behaviour.

