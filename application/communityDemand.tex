% Bonsai has directly enabled high-impact research in neuroscience~\citep[e.g.,][]{soaresEtAl16,grundemannEtAl17,netoEtAl16,hagiharEtAla21,anguillonRodriguezEtAl21} and contributed to accelerating technical progress in the field by functioning as an integrator of open-source, community developed technologies \citep{itskovEtAl14,lopesEtAl21,kaneEtAl20}.
The need for a strong and
flexible ML package that can be easily integrated with
real-time experimental control workflows has been raised multiple times by the
Bonsai user community, both as specific problems around 
tracking and interpretation of animal behaviour, and as 
explicit proposals to create a ML
package\footnote{groups.google.com/g/bonsai-users/c/BZ3zOOdv\_1c/m/x6OP75frBQAJ}. Even
relatively modest contributions such as integrating support for specific models, such as DeepLabCut, or the more specific zebrafish feature-tracking package BonZeb, have attracted positive
feedback from the community, \citep[e.g.,][]{kaneEtAl20,guilbeaultEtAl21}.

The need for an easier interface with ML algorithms
is evidenced by the development of multiple graphical user interfaces for
tracking animal behaviour over the last several years, especially in the much
larger Python user community
\citep[e.g.,][]{walterAndCouzin21,guilbeaultEtAl21}. However, these graphical
user interfaces have no interaction with real-time experimental control, and
their support libraries require expertise with the Python language
to use flexibly. Bonsai addresses this gap, using its
flexible visual programming language to combine ease-of-use and full
flexibility for experimental control.

We focused this proposal on neuroscience applications of Bonsai. However, machine intelligence functionality is needed in many other
experimental areas where Bonsai is or could be used. The software
abstractions that will add ML
functionality to Bonsai for neuroscience experiments (e.g., methods to perform
multi-method comparisons) will also translate to other application domains of
Bonsai.

The ongoing AI revolution makes integration of these
technologies into Bonsai extremely timely. The proposed software infrastructure will facilitate the integration of novel ML methods into the Bonsai
ecosystem. This will unlock new value
and scientific directions from the rich behavioural and neural data that Bonsai already allows scientists to collect. It will also give non-programming experimentalists in the biological
sciences the opportunity to combine real-time ML with
experimental control. To our knowledge
this has not been done before on such a comprehensive scale, and will, we believe, make Bonsai into a fundamental and transformative technology for
the next generation of animal behavioural research.

