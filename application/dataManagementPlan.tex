\documentclass[a4paper,11pt]{article}
\usepackage[margin=2cm]{geometry}
%% Language and font encodings
\usepackage[english]{babel}
\usepackage[utf8x]{inputenc}
\usepackage[T1]{fontenc}
% \usepackage{microtype}      % fine pdf font control

\usepackage[scaled=0.92]{helvet} \renewcommand{\familydefault}{\sfdefault}
\usepackage{xcolor} % -- loaded by tikz


% \usepackage[colorlinks=true]{hyperref}
\usepackage[colorlinks = true,
            linkcolor = blue!50!black!50,
            urlcolor  = blue!50!black!50,
            citecolor = blue!50!black!50,
            anchorcolor = blue!50!black!50]{hyperref}
\usepackage[super]{natbib}
\renewcommand{\citet}[1]{[\citenum{#1}]}
% \usepackage{apalike}
\usepackage{graphicx}
\usepackage{verbatim}

\usepackage{tikz}
\usetikzlibrary{positioning}
\usetikzlibrary{fit}
\usetikzlibrary{calc}

\usepackage{float}
\usepackage{wrapfig}

\usepackage[font={footnotesize}]{caption}

\makeatletter
\renewcommand\paragraph{\@startsection{paragraph}{4}{\z@}%
            {-2.5ex\@plus -1ex \@minus -.25ex}%
            {1.25ex \@plus .25ex}%
            {\normalfont\normalsize\bfseries}}
\makeatother
\setcounter{secnumdepth}{4} % how many sectioning levels to assign numbers to
\setcounter{tocdepth}{4}    % how many sectioning levels to show in ToC
\usepackage[tiny]{titlesec} % format section titles
\def\headfmt{\color{blue!50!black!50}\bfseries}
\titleformat*{\section}{\headfmt}
\titlespacing*{\section}{0pt}{*1}{*0}[-4em]
\titleformat*{\subsection}{\headfmt}
\titlespacing*{\subsection}{0pt}{*0.5}{*0}
\titleformat*{\subsubsection}{\headfmt}
\titlespacing*{\subsubsection}{0pt}{*0}{*0}
\titleformat*{\paragraph}{\headfmt}
\titlespacing*{\paragraph}{0pt}{*0.5}{*0}

\usepackage{enumitem}
\setlist[description]{
    topsep=0pt,
    itemsep=0pt,
    partopsep=0pt,
    parsep=0pt,
    font=\headfmt,
    leftmargin=1em}

\renewcommand{\footnote}[1]{ [#1]}

\title{Data Management Plan}
\author{}
\date{}

\begin{document}

\maketitle

\setcounter{section}{-1}

\section{Proposal name}
% Exactly as in the proposal that the DMP accompanies
Machine intelligence for (neuroscience) experimental control

\section{Description of the data}
\subsection{Type of study}
% Up to three lines of text that summarise the type of study (or studies) for which the data are
% being collected

We propose to enhance an existing software resource, Bonsai, a software
ecosystem to data acquisition and experimental control.

\subsection{Types of data}
% Types of research data to be managed: quantitative or qualitative, sequencing data, images,
% models, software, scripts, protocols, and procedures

Software.

\subsection{Origin of the data}
% Are you creating new primary data (i.e. collecting or generating data)? Or are you re-using
% already available sources, including to create new data (e.g. a new dataset created from
% transformation or integration of existing data)? If data is being re-used, is it publicly accessible?

We will not create primary data for the proposed project. However, as outlined
in \emph{Section 2.5 Testing with neuroscience data} of the \emph{Case for
Support} document, the large community of Bonsai users at the Sainsbuy Wellcome
Centre (SWC) will test the new functionality added to Bonsai prior to
distribution. For these tests, this community of users will use their own
datasets, which will not be made publically available for the purposes of the
current project.

\subsection{Format and scale of the data}
% File formats, software used, number of records, databases, sweeps, repetitions... (in terms that
% are meaningful in your field of research). An indication of the size of data to be stored and
% shared.

\section{Data management, documentation, and curation}
% Keep this section concise and accessible to readers who are not data management experts.
% Focus on principles, systems, and major standards. Focus on the main kind(s) of study data.
% Give brief examples and avoid long lists.
\subsection{Managing, storing and curating data}
% Briefly describe how data will be stored, backed-up, managed and curated in the short to
% medium term. Specify any community agreed or other formal data standards used (with URL
% references).
\subsection{Metadata standards and data documentation}
% What metadata is produced about the data generated from the research? For example
% descriptions of data that enable research data to be used by others outside of your own team.
% This may include documenting the methods used to generate the data, analytical and
% procedural information, capturing instrument metadata alongside data, documenting
% provenance of data and their coding, detailed descriptions for variables, records, etc.

\end{document}
