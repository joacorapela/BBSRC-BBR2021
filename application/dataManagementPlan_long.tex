\documentclass[a4paper,11pt]{article}
\usepackage[margin=2cm]{geometry}
%% Language and font encodings
\usepackage[english]{babel}
\usepackage[utf8x]{inputenc}
\usepackage[T1]{fontenc}
% \usepackage{microtype}      % fine pdf font control

\usepackage[scaled=0.92]{helvet} \renewcommand{\familydefault}{\sfdefault}
\usepackage{xcolor} % -- loaded by tikz


% \usepackage[colorlinks=true]{hyperref}
\usepackage[colorlinks = true,
            linkcolor = blue!50!black!50,
            urlcolor  = blue!50!black!50,
            citecolor = blue!50!black!50,
            anchorcolor = blue!50!black!50]{hyperref}
\usepackage[super]{natbib}
\renewcommand{\citet}[1]{[\citenum{#1}]}
% \usepackage{apalike}
\usepackage{graphicx}
\usepackage{verbatim}

\usepackage{tikz}
\usetikzlibrary{positioning}
\usetikzlibrary{fit}
\usetikzlibrary{calc}

\usepackage{float}
\usepackage{wrapfig}

\usepackage[font={footnotesize}]{caption}

\makeatletter
\renewcommand\paragraph{\@startsection{paragraph}{4}{\z@}%
            {-2.5ex\@plus -1ex \@minus -.25ex}%
            {1.25ex \@plus .25ex}%
            {\normalfont\normalsize\bfseries}}
\makeatother
\setcounter{secnumdepth}{4} % how many sectioning levels to assign numbers to
\setcounter{tocdepth}{4}    % how many sectioning levels to show in ToC
\usepackage[tiny]{titlesec} % format section titles
\def\headfmt{\color{blue!50!black!50}\bfseries}
\titleformat*{\section}{\headfmt}
\titlespacing*{\section}{0pt}{*1}{*0}[-4em]
\titleformat*{\subsection}{\headfmt}
\titlespacing*{\subsection}{0pt}{*0.5}{*0}
\titleformat*{\subsubsection}{\headfmt}
\titlespacing*{\subsubsection}{0pt}{*0}{*0}
\titleformat*{\paragraph}{\headfmt}
\titlespacing*{\paragraph}{0pt}{*0.5}{*0}

\usepackage{enumitem}
\setlist[description]{
    topsep=0pt,
    itemsep=0pt,
    partopsep=0pt,
    parsep=0pt,
    font=\headfmt,
    leftmargin=1em}

\renewcommand{\footnote}[1]{ [#1]}

\title{Data Management Plan}
\author{}
\date{}

\begin{document}

\maketitle

% \textcolor{orange}{\textit{The following template is provided to applicants to assist in the development of a Data Management Plan (DMP) to accompany a research proposal. The notes (in italics) provide further context and guidance for its completion and should be deleted prior to submission. Applicants should consider submission of a DMP which is proportional to the quantity and nature of the data. Where substantial data is generated from the research, it is expected that the DMP will be longer and more detailed than studies generating small amounts of data, where DMPs may be significantly shorter. In either case, the DMP should be a maximum of 3 pages in length.}}

\setcounter{section}{-1}

\section{Proposal name}

% \textcolor{orange}{\textit{Exactly as in the proposal that the DMP accompanies}}

Machine intelligence for (neuroscience) experimental control

\section{Description of the data}
\subsection{Type of study}

% \textcolor{orange}{\textit{Up to three lines of text that summarise the type of study (or studies) for which the data are being collected}}

The project will generate software modules appropriate for use in behavioural neuroscience experiments in animal models, and other types of study that involve a combination of reactive experimental (particularly behavioural) control and streamed data acquisition.  

% Enhancement of an existing software resource, the Bonsai software
% ecosystem for data acquisition and experimental control.

\subsection{Types of data}

% \textcolor{orange}{\textit{Types of research data to be managed: quantitative or
% qualitative, sequencing data, images, models, software, scripts, protocols, and
% procedures}}

This is a software project.  We will produce and manage new modules to be integrated into the Bonsai software ecosystem.  These will include communication modules to be written in C\#, and machine-learning applications written in Python, R and Matlab.

% We propose to incorporate into the Bonsai software ecosystem state-of-the-art
% machine-learning models, as well as capabilities to enable the communication
% between Bonsai, written in C\#, and machine-learning applications written in
% Python, R and Matlab.

\subsection{Origin of the data}

% \textcolor{orange}{\textit{Are you creating new primary data (i.e. collecting or
% generating data)? Or are you re-using already available sources, including to
% create new data (e.g. a new dataset created from transformation or integration
% of existing data)? If data is being re-used, is it publicly accessible?}}

New modules will be created and tested by software research engineers at UCL in consultation with staff at NeuroGEARS Ltd (a project collaborator).  These will be integrated into the existing Bonsai software ecosystem.

We will access experimental data during the testing phase, but will not seek to manage these data within the project.  Such data will have been collected in the course of other projects and will continue to be managed by the relevant laboratories in accordance with the protocols they have defined. 

% We will not create primary data for the proposed project. However, as outlined
% in \emph{Section 2.5 Testing with neuroscience data} of the \emph{Case for
% Support} document, the large community of Bonsai users at the Sainsbuy Wellcome
% Centre (SWC) will test the new functionality added to Bonsai prior to its
% distribution. For these tests, this community will use their own datasets, its
% including high-density Neuropixels physiological recordings, with optogentic
% manipulations and high-resolution video monitoring of animal behaviour. These
% datasets will not be made publicly available for the purposes of the proposed
% project. However, we are building infrastructure to make them available in the
% near future.

% The proposed enhancements to the Bonsai ecosystem will impact a wide range of
% existing and to-be-created datasets collected by open-source initiatives
% already integrated into the Bonsai ecosystem. For example, datasets of feeding
% behaviour in flies (collected with FlyPAD
% sensors\footnote{\url{https://flypad.rocks/}}), high-resolution zebrafish
% tracking datasets (collected with
% BonZeb\footnote{\url{https://github.com/ncguilbeault/BonZeb/}}),
% electrophysiology datasets (collected with products supported by
% OpenEphys\footnote{\url{https://open-ephys.org/}} or with NeuroPixels
% probes\footnote{\url{https://www.neuropixels.org/}})), datasets of in-vivo
% imaging of brain activity (collected with the UCLA
% Miniscope\footnote{\url{http://miniscope.org/}} or with NeuroPhotometrics
% products\footnote{\url{https://neurophotometrics.com/}}), datasets using rich
% 2D or 3D visual environments (created with
% BonVision\footnote{\url{https://bonvision.github.io/}}), and animal behavioral
% datasets (with body parts tracked with
% DeepLabCut\footnote{\url{http://www.mackenziemathislab.org/deeplabcut}}).

\subsection{Format and scale of the data}

% \textcolor{orange}{\textit{File formats, software used, number of records, databases, sweeps, repetitions... (in terms that are meaningful in your field of research). An indication of the size of data to be stored and shared.}}

Software modules will comprise source files in various programming languages: C\#, Python, MATLAB and R.  These will be accompanied by extensive documentation (discussed below).

**NEED**: Approximate lines of code??? (order of magnitude is fine)

**NEED**: testing files, makefiles?, other stuff, 


% The source code for the core Bonsai ecosystem and the majority of extension
% packages are hosted at \url{github.com/bonsai-rx}.

\section{Data management, documentation, and curation}

% \textcolor{orange}{\textit{Keep this section concise and accessible to readers who are not data management experts.  Focus on principles, systems, and major standards. Focus on the main kind(s) of study data.  Give brief examples and avoid long lists.}}


\subsection{Managing, storing and curating data}

% Briefly describe how data will be stored, backed-up, managed and curated in the short to medium term. Specify any community agreed or other formal data standards used (with URL references). 

Software will be developed within the git distributed version control system, with public repositories made available within a specially created organisation at github.com.  

Copies of all repositories will be maintained within the Sainsbury Wellcome Centre data storage system at UCL.  **NEED**: short/medium term backup protocols from John Pelan's DMP

% management
Package installation and updates are supported by standalone package management
software included within the core Bonsai distribution and hosted in a shared
centralised and curated repository. This ensures that bugfixes and enhancements
are easily available and, importantly, allows for reproducible deployment of
user environments to ensure working experiments are not disrupted by system
changes.
%
Large-scale experimental projects such as the International Brain Laboratory
have leveraged this capability of Bonsai to standardise data acquisition and
control software across dozens of laboratories around the world.
%
Community technology previews are made available through the package manager,
letting early adopters opt-in to the latest features before they are generally
available. This allows the community to report issues with new features on
GitHub, thereby distributing bug reporting and triage, and improving the
quality of stable releases.

The core compiler and IDE is extensively tested, and uses a strong type system
to greatly reduce bugs at compile-time. Best practices for leveraging this
infrastructure for building new packages are in place. The software engineers
recruited to this project will work with our collaborators at NeuroGEARS Ltd,
to strengthen and maintain these practices. Once fully familiar with the Bonsai
core architecture they will participate in review of pull requests and other
code contributions.


% Bonsai is a freely available open-source resource, already widely adopted by
% the systems neuroscience and behavioural science communities (see Fig 1f).
% Current contributors include other large open projects (OpenEphys, DeepLabCut),
% and companies developing and supplying new hardware devices
% (Neurophotometrics). This community uptake, engagement, and commitment provides
% a foundation for long-term sustainability.

% NeuroGEARS Ltd.\ is a keystone of this community and a key collaborator in the
% current proposal. Founded by the original developer of Bonsai, it helps to
% support and expand the user base well beyond the life of the current proposal.
% NeuroGEARS currently provides quality control and curation for the core Bonsai
% code, and provides training and support services for research institutions
% around the world who want to start using Bonsai. NeuroGEARS also facilitates
% knowledge transfer between users and institutions adopting Bonsai, by
% moderating the community user forums and collaborating with organisations who
% want to build new innovative research projects on top of Bonsai, as in the
% current proposal.

% The specific contributions developed as part of this project will form new core
% components and packages within the Bonsai ecosystem. In particular, the new
% language integration packages will support long-term sustainability, by
% dramatically increasing Bonsai's interoperability with existing software
% development communities which are already well established in basic science
% (MATLAB, Python, R). Developing these new packages with input from the core
% advisory panel of Bonsai, and early public distribution and user testing within
% the SWC and with planned workshops, will ensure solid and broad user engagement
% from the beginning of the development cycle and will therefore build momentum
% for their use, support and dissemination. Close collaboration with the core
% maintainers of the Bonsai ecosystem at NeuroGEARS will allow them to take over
% responsibility for maintenance and support of these packages beyond the end of
% the project, following the same principles of open-source and free sharing of
% knowledge that characterises the entire Bonsai ecosystem.

\subsection{Metadata standards and data documentation}

% \textcolor{orange}{\textit{What metadata is produced about the data generated from the research? For example descriptions of data that enable research data to be used by others outside of your own team.  This may include documenting the methods used to generate the data, analytical and procedural information, capturing instrument metadata alongside data, documenting provenance of data and their coding, detailed descriptions for variables, records, etc.}}

% documentation
Current Bonsai documentation, including learning materials and all relevant community
links are publicly accessible at \url{bonsai-rx.org}.

Documentation will be created for the new modules produced in this project that meets the existing standards, and will be integrated into the Bonsai documentation framework.

% The fact that Bonsai is distributed as an open source project
% (\url{https://bonsai-rx.org/}), that is has extensive online documentation
% (\url{https://bonsai-rx.org/docs/}) and a vibrant user discussion forum
% (\url{https://groups.google.com/g/bonsai-users}), and that is integrated with
% many other open-source projects (Figure~1A and~1B, \emph{Case for Support})
% enables the use of the Bonsai ecosystem by the general public.


\subsection{Data preservation strategy and standards}

% Plans and place for long-term storage, preservation, and planned retention period for the research data. Formal preservation standards, if any. Indicate which data may not be retained (if any). Also, briefly describe what will be stored long-term, raw data, processed data?

\section{Data sharing and access}
\subsection{Where will data be shared?}
% Identify any data repository (-ies) that are, or will be, entrusted with storing, curating and/or sharing data from your study, where they exist for particular disciplinary domains or data types.
Software modules and documentation will be available through the github.com service, with links from the Bonsai project site (https://bonsai-rx.org) and available through the Bonsai system's package management interface.  

**NEED**: github org and status?

Repositories will be mirrored within the Sainsbury Wellcome Centre scientific computing system.  **NEED**: long-term backup from John's DMP 


\subsection{When will data be available?}
% It is expected that timely release would generally be no later than the release through publication of the main findings and should be in line with established best practice in the field. Where best practices do not exist release within three years of generation of the dataset is suggested as a guide.

Release management: repositories will be public throughout, but marked as developmental.  Alpha-level code will be available but not publicised until after extensive testing.  Will be integrated into package management and advertised once it achieves release quality.

\subsection{How will data be made findable and accessible?}

% Indicate how potential new users (outside of your organisation) can find out about your data and identify whether it could be suitable for their research purposes, also consider how it will be licenced. The approach should encompass data in all its forms as well as software. This section could also outline access procedures, how your data will be able to be cited and tracked.

Package management
Bonsai website
Community fora
Workshops
Paper?
 A forum where Bonsai
users daily exchange questions and answers can be accessed at
\url{https://groups.google.com/g/bonsai-users}. Currently more than 700 users
contribute to this forum (Figure~1f, \emph{Case for Support}).

\subsection{How will data be made reusable?}

% What steps will you take to ensure that your data could be re-analysed by other researchers? (e.g. availability of meta data, worked examples, vignettes). How will metadata be made available?

This is an open source software project.  Code will be distributed under **NEED** licence, and thus will be freely reusable and modifiable.



\subsection{Restrictions or delays to sharing, with planned actions to limit such restrictions} 
% Restriction to data sharing may be due to subject confidentiality or to ensure you can gain IP protection. Applicants should ensure they have obtained necessary clearances from relevant collaborators with regards to the content of the proposal including the data sharing plan in line with the BBSRC Research Grants Guide. Where possible please indicate when data will be made available following a period of restriction.  

All distributed software will be developed as part of the project, based on published and unrestricted algorithms.  Thus it is unencumbered by IP restrictions and will be available throughout development as described above.

\section{Data security (where relevant)}

The software produced will be freely available and open sourced.  It will contain no personal data.  User access to the repository will be anonymous and user personal information will not be collected.  Thus security concerns do not apply.


\end{document}
