\documentclass[a4paper,11pt]{article}
\usepackage[margin=2cm]{geometry}
%% Language and font encodings
\usepackage[english]{babel}
\usepackage[utf8x]{inputenc}
\usepackage[T1]{fontenc}
% \usepackage{microtype}      % fine pdf font control

\usepackage[scaled=0.92]{helvet} \renewcommand{\familydefault}{\sfdefault}
\usepackage{xcolor} % -- loaded by tikz


% \usepackage[colorlinks=true]{hyperref}
\usepackage[colorlinks = true,
            linkcolor = blue!50!black!50,
            urlcolor  = blue!50!black!50,
            citecolor = blue!50!black!50,
            anchorcolor = blue!50!black!50]{hyperref}
% \usepackage[super]{natbib}
% \renewcommand{\citet}[1]{[\citenum{#1}]}
\usepackage{natbib}
\usepackage{apalike}
\usepackage{graphicx}
\usepackage{verbatim}

\usepackage{tikz}
\usetikzlibrary{positioning}
\usetikzlibrary{fit}
\usetikzlibrary{calc}

\usepackage{float}
\usepackage{wrapfig}

\usepackage[font={footnotesize}]{caption}

\makeatletter
\renewcommand\paragraph{\@startsection{paragraph}{4}{\z@}%
            {-2.5ex\@plus -1ex \@minus -.25ex}%
            {1.25ex \@plus .25ex}%
            {\normalfont\normalsize\bfseries}}
\makeatother
\setcounter{secnumdepth}{4} % how many sectioning levels to assign numbers to
\setcounter{tocdepth}{4}    % how many sectioning levels to show in ToC
\usepackage[tiny]{titlesec} % format section titles
\def\headfmt{\color{blue!50!black!50}\bfseries}
\titleformat*{\section}{\headfmt}
\titlespacing*{\section}{0pt}{*1}{*0}[-4em]
\titleformat*{\subsection}{\headfmt}
\titlespacing*{\subsection}{0pt}{*0.5}{*0}
\titleformat*{\subsubsection}{\headfmt}
\titlespacing*{\subsubsection}{0pt}{*0}{*0}
\titleformat*{\paragraph}{\headfmt}
\titlespacing*{\paragraph}{0pt}{*0.5}{*0}

\usepackage{enumitem}
\setlist[description]{
    topsep=0pt,
    itemsep=0pt,
    partopsep=0pt,
    parsep=0pt,
    font=\headfmt,
    leftmargin=1em}

\renewcommand{\footnote}[1]{ [#1]}

\title{Data Management Plan}
\author{}
\date{}

\begin{document}

\maketitle

\textcolor{orange}{\textit{The following template is provided to applicants to assist in the development of a Data Management Plan (DMP) to accompany a research proposal. The notes (in italics) provide further context and guidance for its completion and should be deleted prior to submission. Applicants should consider submission of a DMP which is proportional to the quantity and nature of the data. Where substantial data is generated from the research, it is expected that the DMP will be longer and more detailed than studies generating small amounts of data, where DMPs may be significantly shorter. In either case, the DMP should be a maximum of 3 pages in length.}}

\setcounter{section}{-1}

\section{Proposal name}

\textcolor{orange}{\textit{Exactly as in the proposal that the DMP accompanies}}

Machine intelligence for (neuroscience) experimental control

\section{Description of the data}
\subsection{Type of study}

\textcolor{orange}{\textit{Up to three lines of text that summarise the type of study (or studies) for which the data are being collected}}

No primary data are being collected for the proposed software resource.

\subsection{Types of data}

\textcolor{orange}{\textit{Types of research data to be managed: quantitative or
qualitative, sequencing data, images, models, software, scripts, protocols, and
procedures}}

We propose to manage software and models. We offer to incorporate into the
Bonsai software ecosystem state-of-the-art machine-learning models, as well as
capabilities to enable the communication between Bonsai, written in C\#, and
machine-learning applications written in Python, R and Matlab.

\subsection{Origin of the data}

\textcolor{orange}{\textit{Are you creating new primary data (i.e. collecting or
generating data)? Or are you re-using already available sources, including to
create new data (e.g. a new dataset created from transformation or integration
of existing data)? If data is being re-used, is it publicly accessible?}}

We will not create primary data for the proposed project. However, as outlined
in \emph{Section 2.5 Testing with neuroscience data} of the \emph{Case for
Support} document, the large community of Bonsai users at the Sainsbuy Wellcome
Centre (SWC) will test the new functionality added to Bonsai prior to its
distribution. For these tests, this community will use their own datasets, its
including high-density Neuropixels physiological recordings, with optogentic
manipulations and high-resolution video monitoring of animal behaviour. These
datasets will not be made publicly available for the purposes of the proposed
project. However, we are building infrastructure to make them available in the
near future.

The proposed enhancements to the Bonsai ecosystem will impact a wide range of
existing and to-be-created datasets collected by open-source initiatives
already integrated into the Bonsai ecosystem. For example, datasets of feeding
behaviour in flies (collected with FlyPAD
sensors\footnote{\url{https://flypad.rocks/}}), high-resolution zebrafish
tracking datasets (collected with
BonZeb\footnote{\url{https://github.com/ncguilbeault/BonZeb/}}),
electrophysiology datasets (collected with products supported by
OpenEphys\footnote{\url{https://open-ephys.org/}} or with NeuroPixels
probes\footnote{\url{https://www.neuropixels.org/}})), datasets of in-vivo
imaging of brain activity (collected with the UCLA
Miniscope\footnote{\url{http://miniscope.org/}} or with NeuroPhotometrics
products\footnote{\url{https://neurophotometrics.com/}}), datasets using rich
2D or 3D visual environments (created with
BonVision\footnote{\url{https://bonvision.github.io/}}), and animal behavioral
datasets (with body parts tracked with
DeepLabCut\footnote{\url{http://www.mackenziemathislab.org/deeplabcut}}).

\subsection{Format and scale of the data}

\textcolor{orange}{\textit{File formats, software used, number of records, databases, sweeps, repetitions... (in terms that are meaningful in your field of research). An indication of the size of data to be stored and shared.}}

We believe this section does not applies to our proposed software resource that
will not generate primary data.

\section{Data management, documentation, and curation}

\textcolor{orange}{\textit{Keep this section concise and accessible to readers who are not data management experts.  Focus on principles, systems, and major standards. Focus on the main kind(s) of study data.  Give brief examples and avoid long lists.}}

Please refer to \emph{Section 5 Resource management} of the \emph{Case for
support} document for details about how we plan to manage the proposed resource
development.

\subsection{Managing, storing and curating data}

\textcolor{orange}{\textit{Briefly describe how data will be stored, backed-up, managed and curated in the short to medium term. Specify any community agreed or other formal data standards used (with URL references).}}

Please refer to \emph{Section 6 Long-term sustainability planning} of the
\emph{Case for support} document for details about how we plan the long-term
sustainability of the proposed resource development.

\subsection{Metadata standards and data documentation}

\textcolor{orange}{\textit{What metadata is produced about the data generated from the research? For example descriptions of data that enable research data to be used by others outside of your own team.  This may include documenting the methods used to generate the data, analytical and procedural information, capturing instrument metadata alongside data, documenting provenance of data and their coding, detailed descriptions for variables, records, etc.}}

We believe this section does not applies to our proposed software resource that
will not generate primary data.

% \bibliographystyle{unsrtveryabbrv}
\bibliographystyle{apalike}
\bibliography{bonsai}

\end{document}
