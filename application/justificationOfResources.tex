\documentclass[a4paper,11pt]{article}
\usepackage[margin=2cm]{geometry}

\title{Justification of Resources}
\author{}
\date{}

\begin{document}

\maketitle

\section{Directly Incurred - Staff} 

\subsection*{Research software engineers (RSE)}

To implement the proposed project we will recruit two experienced RSEs, appointed
at grade 8. They will be responsible for developing and implementing all the
work packages throughout the project. We anticipate one of the RSEs to be Dr.
Joaquin Rapela, who will continue to work on integrating machine-learning
behavioural and neural analysis tools with Bonsai, in collaboration with
NeuroGEARS. He has the necessary machine learning and programming expertise to
deliver the proposed work. An additional RSE with a broad skillset (programming
in C\#, Python, R, Matlab, Windows) is required to deliver foreign programming language interface packages. They will work closely with the software engineering team at NeuroGEARS. Total cost to funder: £196,864 for each of the two RSEs. 

\section{Directly Incurred - Travel \& Subsistence} 

\subsection*{Conferences}

The success of the proposed project will depend on continued training and community engagement of employed RSEs, to stay abreast of recent developments in machine learning and neural and behavioural analysis. Conferences are key resources for such knowledge acquisition. We request funding to support the attendance of at least one
conference per year for each RSE. While many conferences now offer online formats, others are returning to in-person attendance, which is crucial for networking and broadening Bonsai user engagement. Possible conferences that could benefit the RSEs include SciPy, DotNet Conference, PyCon, Matlabexpo, UseR!, and SeptembRSE Conferences for Research Software Engineers. 

Costs include travel and accommodation for 3 years for each RSE. Total cost to funder: £11,520 (£5,760 per RSE).

\subsection*{Networking Event}

We request funds for a user engagement event to mark the 10th anniversary of Bonsai. This small 3-day conference will have approximately 80-100 attendees and about 6 invited international scientists or developers who use or develop the Bonsai ecosystem. The conference will have 1 day of talks, followed by a separate 2-day hackathon (see below). The intention of this conference is to promote awareness of our software and bring together the community that uses our software to discuss possible future directions, in addition to bringing in new users. The conference would be a hybrid event with an online component for Bonsai users across the globe to also attend virtually. We request funds for travel and accommodation of invited speakers, in addition to catering for lunch and coffee breaks. We also request funds for an online platform for the conference. Travel and accommodation cost for speakers: £10,560. Catering and online platform costs: £1,920.

\subsection{Workshops} 

We request funds for two Bonsai workshops to take place at the Sainsbury Wellcome Centre. The aim of the workshops is to increase dissemination and user engagement. 

Bonsai for Software Developers Workshop. This will be a training course and workshop focusing explicitly on the extensibility features of the Bonsai programming languages, targeting software developers interested in making their own Bonsai packages and custom language integration. This workshop would take place at the end of year 2, towards the end of the development of the software infrastructure, as a way of presenting the development proof-of-concept to the community and possibly bringing new developers from the community to contribute to the resource from the outset.

Machine Learning in Bonsai Hackathon. This would be a collaborative workshop potentially combining scientists and engineers interested in developing applications of machine learning in Bonsai, either to address scientific questions, or to integrate new methods and algorithms into the resource (these could be developed as two alternate/complementary tracks running in parallel for participants). We expect to integrate this workshop with the user engagement event described above.

Each workshop will run over the course of two days. Registration for each workshop will be free of charge. We request funds for catering. Total cost for the two workshops: £1920.

\section{Other Directly Incurred} 

\subsection*{Equipment}

We request funds for two high-end laptop computers to facilitate possible remote working for each RSE. Total cost to funder: £4,800 (£2400 per RSE).  

\subsection*{Training}

To extend the knowledge base of RSEs working on this project, they may benefit from formal training courses in software design, programming, machine learning and/or computational neuroscience. Total cost of sign-up fees: £1,920. 

\subsection*{Conference fees}

We request funds to cover the registration fees for the two RSEs to attend at least 3 conferences for the duration of the project.  
Total cost to funder: £2,112 (£1,056 per RSE). 

\end{document}
