% Instead of submitting a traditional academic CV, we invite applicants to submit a two-page 
% Narrative CV. This should act as a narrative which explains how the individual’s relevant 
% experience and expertise demonstrates their ability to successfully deliver the proposal. A 
% CV which simply lists past positions, publications, and funding will not adequately support an 
% application.
% 
% 
% Applicants should draw on a breadth of examples which illustrate of they how they have 
% contributed to new ideas, hypotheses and tools, as well as how they have contributed to 
% teams and collaborations, the research community, and to wider society. Applicants should 
% describe only a selection of their outputs and, in each case, clearly explain the relevance to 
% their ability to deliver the proposed project.
% 
% 
% All individuals should describe contributions across the levels below. The relative size 
% of each section will vary depending on the relevant skills and
% experience of each applicant.
% 
%

\section{Dr.\ Joaquin Rapela -- narrative CV}


% 1) Eligibility criteria: State your current position, indicating how you meet the 
% eligibility criteria as outlined within the UKRI-BBSRC Grants Guide.

\subsection{Eligibility criteria}

Dr.~Joaquin Rapela is a Research Engineer Fellow at the Gatsby Computational
Neuroscience Unit (GCNU), University College London.
%
He has made a substantial contribution to the formulation and development of
the proposed project, and will be engaged with the ensuing research. He is a
Research Co-Investigator in the current application.

% 2) How have you contributed to the generation and flow of new ideas, hypotheses, 
% tools or knowledge?
% 
% Examples might include: contributions to and skills acquired from past research projects, and 
% key outputs such as data sets, software, and research and policy publications. In each case 
% the relevance to delivering the proposed project should be summarised.

\subsection{Contributions to ideas}

Dr.~Rapela has acquired unique skills for the successful implementation of the
proposed project. He completed his undergraduate degree in Computer Sciences,
worked as a staff software engineer at the IBM Almaden Research Center, before
pursuing his doctoral degree in Electrical Engineer, with specialisation on
Signal Processing, at the University of Southern California.  His doctoral
dissertation, as well as his two postdoctoral positions at University of
California San Diego (UCSD) and at Brown University, focused on the use of
advances mathematical and statistical tools to characterise
electrophysiological neural signals. His
publications\footnote{https://scholar.google.com/citations?user=eXkDg2UAAAAJ\&hl=en}
demonstrate unique skills in signal processing, statistics and computational
neuroscience.

Dr.~Rapela is a strong supporter of open source software, and all his
publications are accompanied by free
code\footnote{http://www.gatsby.ucl.ac.uk/~rapela/software.htm}. While at UCSD,
he contributed to the development of EEGLAB, a large open-source project for
the characterisation of the electroencephalogram. Currently, as an Engineer
Research Fellow at the GCNU, he distributes high-quality implementations of
algorithms developed at unit (e.g., sparse variational Gaussian process factor
analysis\footnote{https://github.com/joacorapela/svGPFA}).

Another key responsibility of Dr.~Rapela at the GCNU, is to establish
collaborations with experimental neuroscientists at the SWC and help them on
the use of advanced statistical methods to process their state-of-the-art
neural recordings.

Therefore, Dr.~Rapela's expertise is perfectly suited to implement all aspects
of the
proposed project (neuroscience, statistics, signal processing, and software development).

% 3) How have you contributed to research teams and the development of
% others? % Examples might include: project management, supervision, mentoring or line management 
% contributions critical to the success of a team or team members, or where you exerted 
% strategic leadership in shaping the direction of a team, organisation, company or institution.

\subsection{Contributions to the development of others}

Dr.~Rapela has mentored Mr.~Tsong-Yan Lin (master student at University
California San Diego) on signal processing for processing
electroencephalographic and electrooculographic time
series~\footnote{Joaquin Rapela, Tsong-Yan Lin, Marissa Westerfield, Tzyy-Ping
Jung, and Jeanne Townsend (2012b). Assisting autistic children with wireless
EOG technology. Proceedings of the 34th Annual International Conference of the
IEEE EMBS, San Diego, California.}.

Currently, Dr.~Rapela is mentoring Ms.~Aishah Qureshi (undergraduate student at
Queens College, London), as part of the Simons Foundation Undergraduate Research
Program, on the use of linear dynamical systems to understand long-range
cortical communication in rodents.

% \subsection{Contributions to teams and development}

% 4) How have you contributed to the wider research community?
% 
% Examples might include: how you have contributed to wider collaborations and networks 
% across disciplines, institutions, and / or countries, commitments such as editing, reviewing 
% and committee work, positions of responsibility, and activities which have contributed to the 
% improvement of research integrity or culture, or examples where you have shown visionary 
% strategic leadership in influencing a research agenda.  

\subsection{Contributions to wider research community}

Dr.~Rapela has reviewed articles for Vision Research, Frontiers in Neuroscience
and the Journal of Perceptual Imaging. He currently contributes to the SWC
Research Culture Working
Group~\footnote{https://sainsburywellcomecentre.github.io/RCWG/research} to
improve the research culture at the SWC and GCNU.

% 5) How have you contributed to broader society? 
% Examples might include: engagement across the public and / or private sectors or with the 
% wider public, past research which has contributed to policy development or public 
% understanding, and other impacts across research, policy, practice and business, and other 
% examples of and how you have ensured your research reaches and influences relevant 
% audiences. 



% 6) Additional information  
% Any additional relevant information you wish to include in support of your capability to deliver, 
% which may include further information about key qualifications and relevant positions, 
% secondments, volunteering, or other relevant experience such as time spent in different 
% sectors. There is no need to provide information about career breaks, part-time working etc, 
% however if there are any details you do wish panel members to consider in their assessment 
% of the proposal they may be included here.
% 
