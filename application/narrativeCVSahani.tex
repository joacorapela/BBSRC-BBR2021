% Instead of submitting a traditional academic CV, we invite applicants to submit a two-page 
% Narrative CV. This should act as a narrative which explains how the individual’s relevant 
% experience and expertise demonstrates their ability to successfully deliver the proposal. A 
% CV which simply lists past positions, publications, and funding will not adequately support an 
% application.
% 
% 
% Applicants should draw on a breadth of examples which illustrate of they how they have 
% contributed to new ideas, hypotheses and tools, as well as how they have contributed to 
% teams and collaborations, the research community, and to wider society. Applicants should 
% describe only a selection of their outputs and, in each case, clearly explain the relevance to 
% their ability to deliver the proposed project.
% 
% 
% All individuals should describe contributions across the levels below. The relative size 
% of each section will vary depending on the relevant skills and
% experience of each applicant.
% 
%

\section{Prof.\ Maneesh Sahani -- narrative CV}


% 1) Eligibility criteria: State your current position, indicating how you meet the 
% eligibility criteria as outlined within the UKRI-BBSRC Grants Guide.

\subsection{Eligibility criteria}

Prof.\ Sahani is a Professor of Theoretical Neuroscience and Machine
Learning, and director of the Gatsby Computational Neuroscience Unit
at UCL.
%
As such, he is resident in the UK, and holds an open-ended academic
employment contract with UCL.

% 2) How have you contributed to the generation and flow of new ideas, hypotheses, 
% tools or knowledge?
% 
% Examples might include: contributions to and skills acquired from past research projects, and 
% key outputs such as data sets, software, and research and policy publications. In each case 
% the relevance to delivering the proposed project should be summarised.

\subsection{Contributions to ideas}

Prof.\ Sahani has authored over 100 peer-reviewed scientific papers,
with an h-index (computed by Google scholar) of 47.

GPFA

PLDS

Grant funding?

% 3) How have you contributed to research teams and the development of
% others?

% From QQR:
As a PI, I have supervised a total of \textbf{X} students and \textbf{X} postdocs.  
In the last five years, 4 of students have moved to postdoctoral positions (at
Aalto, Cambridge, CMU and Stanford), 1 works at DeepMind, and 1 holds a faculty offer at Shanghai Tech.  In all, 8 previous students have now secured faculty (or group leader) positions at Cambridge, CMU, Columbia, Copenhagen, Janelia (2), Pittsburgh and Shanghai.
%
Of the postdoctoral fellows I have supervised in the last five years, 1 is doing further postdoc work in Bonn; 1 holds an independent
Research Fellowship at Radboud; and 2 have faculty positions: at Edinburgh and at UBC.

DARPA repair

Gatsby directorship
 
% Examples might include: project management, supervision, mentoring or line management 
% contributions critical to the success of a team or team members, or where you exerted 
% strategic leadership in shaping the direction of a team, organisation, company or institution.  


% 4) How have you contributed to the wider research community?
% 
% Examples might include: how you have contributed to wider collaborations and networks 
% across disciplines, institutions, and / or countries, commitments such as editing, reviewing 
% and committee work, positions of responsibility, and activities which have contributed to the 
% improvement of research integrity or culture, or examples where you have shown visionary 
% strategic leadership in influencing a research agenda.  

I have been an active member of the academic community. I co-founded the Neural Coding, Computation and Dynamics meeting in 2007, and was in the organising committee in 2015 and 2017. 

I participated in the selection committee for the Swartz Prize for Theoretical and Computational Neuroscience in 2014-2016, and was in the programme committee for the Computational and Systems Neuroscience (COSYNE) conference in 2007, becoming a  Programme Chair in 2009 and General Chair in 2010. 

I was also in the Programme Committee for NIPS (now NeurIPS) in 2004 and 2006, becoming a Workshops Chair in 2008. I was also a Workshops Chair for the Computational Neuroscience Meeting from 1999 - 2003. 

I was a member of the Board of Directors of the Computational Neuroscience Organization between 2003 - 2006. I am currently a member of the Society for Neuroscience and IEEE, and in 2021 became a Fellow of the European Laboratory for Learning and Intelligent Systems (ELLIS). 

I currently have editorial roles in various journals including, Current Opinion in Neurobiology (Special issue on Big Data and Neuroscience), Neural Computation, Neural Systems and Circuits, Network:
Computation in Neural Systems, and Faculty of
1000, "Machine learning: life sciences" collection. 

I have taken various advisory roles for international review and award panels. I reviewed the Donders Institute for Brain, Cognition and Behaviour (Netherlands), The Edmond and Lily Safra Center for Brain Sciences, Hebrew
University (Israel), and the Barcelona Summer School on Advanced Modelling of Behaviour (Spain). I have advised on awards such as the Swartz Prize for Theoretical and Computational Neuroscience (SFN, US), the Bernstein Prize (BMBF, Germany), and COSYNE.

Recently, I have acted as an advisor to a new established cross-disciplinary effort at the Indian Institute of Science in Bangalore; visiting the site a
number of times and hosting a joint workshop here at UCL.

I have sat on various grant award panels for the BMBF and DFG, and the (US) NSF/NIH collaborative research
in computational neuroscience (CRCNS) programme. I have also acted as a reviewer for proposals for the Bernstein prize, Einstein Foundation, EPSRC, MRC, Novo Nordisk Foundation (Denmark) and NWO (Netherlands).

% 5) How have you contributed to broader society? 
% Examples might include: engagement across the public and / or private sectors or with the 
% wider public, past research which has contributed to policy development or public 
% understanding, and other impacts across research, policy, practice and business, and other 
% examples of and how you have ensured your research reaches and influences relevant 
% audiences. 



% 6) Additional information  
% Any additional relevant information you wish to include in support of your capability to deliver, 
% which may include further information about key qualifications and relevant positions, 
% secondments, volunteering, or other relevant experience such as time spent in different 
% sectors. There is no need to provide information about career breaks, part-time working etc, 
% however if there are any details you do wish panel members to consider in their assessment 
% of the proposal they may be included here.
% 
