% Instead of submitting a traditional academic CV, we invite applicants to submit a two-page 
% Narrative CV. This should act as a narrative which explains how the individual’s relevant 
% experience and expertise demonstrates their ability to successfully deliver the proposal. A 
% CV which simply lists past positions, publications, and funding will not adequately support an 
% application.
% 
% 
% Applicants should draw on a breadth of examples which illustrate of they how they have 
% contributed to new ideas, hypotheses and tools, as well as how they have contributed to 
% teams and collaborations, the research community, and to wider society. Applicants should 
% describe only a selection of their outputs and, in each case, clearly explain the relevance to 
% their ability to deliver the proposed project.
% 
% 
% All individuals should describe contributions across the levels below. The relative size 
% of each section will vary depending on the relevant skills and
% experience of each applicant.
% 
%

{
  \parindent 0pt
  \parskip 1ex

\section{Prof.\ Maneesh Sahani -- narrative CV}


% 1) Eligibility criteria: State your current position, indicating how you meet the 
% eligibility criteria as outlined within the UKRI-BBSRC Grants Guide.

\subsection{Academic history and eligibility criteria}

I am currently \textbf{Professor of Theoretical Neuroscience and Machine Learning}, and
director of the Gatsby Computational Neuroscience Unit at UCL.
%
I am a UK resident and hold an open-ended academic employment contract
with UCL.

I earned my PhD from the California Institute of Technology (Caltech)
in 1999, working in the laboratories of Profs.\ Richard Andersen and
John Hopfield.
%
After post-doctoral work at UCL/Gatsby (with Prof.\ Peter Dayan) and
UCSF (Prof.\ Michael Merzenich) I joined the academic staff of the
Gatsby Unit as a Lecturer in 2004, becoming Reader in 2009 and
Professor in 2013.
%
I took over the Directorship of the Unit at the end of 2017.

% 2) How have you contributed to the generation and flow of new ideas, hypotheses, 
% tools or knowledge?
% 
% Examples might include: contributions to and skills acquired from past research projects, and 
% key outputs such as data sets, software, and research and policy publications. In each case 
% the relevance to delivering the proposed project should be summarised.

\subsection{Contributions to ideas, hypotheses, tools and knowledge}

I have authored over 100 peer-reviewed scientific papers, with an
h-index (computed by Google scholar) of 47.

A substantial component of my research focuses on the development of
advanced machine-learning based tools for neuroscience research.
%
This included my PhD research.  My thesis \textit{Latent variable
  models for neural data analysis} introduced new techniques of
probabilistic inference and applied these to the problem of ``spike
sorting'' and clustering of neuronal activity profiles.  It has been cited
over 180 times.


Beginning around 2005, my group published a series of new
neuroinformatics tools designed to characterise and understand
population-scale activity using the large-scale multielectrode
recording methods being developed.  These papers provided the backbone
for a new analytic approach that is now being employed and extended by
systems neuroscience laboratories worldwide.
%
A central component of the current proposal is to make this approach
(and others) available within Bonsai, easing its adoption by a wider
group of laboratories that lack in-house informatics expertise.
%
Key foundational papers include:
\begin{itemize}[topsep=0pt,itemsep=0pt]
\item  B. M. Yu, A. Afshar, G. Santhanam, S. I. Ryu, K. V. Shenoy, and M. Sahani.
Extracting dynamical structure embedded in neural activity.
In Y. Weiss, B. Schölkopf, and J. Platt, eds., Advances in Neural
Information Processing Systems, vol. 18, pp. 1545–1552. MIT Press,
Cambridge, Massachusetts, 2006.\\
Cited 91 times.
% \item M. M. Churchland, B. M. Yu, M. Sahani, and K. V. Shenoy.
% Techniques for extracting single-trial activity patterns from large-scale neural recordings.
% Current Opinion in Neurobiology, 17(5):609–618, 2007.\\
% Review, cited 183 times.
\item B. M. Yu, J. P. Cunningham, G. Santhanam, S. I. Ryu, K. V. Shenoy, and M. Sahani.
Gaussian-process factor analysis for low-dimensional single-trial analysis of neural population activity.
Journal of Neurophysiology, 102:614–635, 2009.\\
Introduced the GPFA method; cited 561 times.
% \item B. Petreska, B. M. Yu, J. P. Cunningham, G. Santhanam, S. I. Ryu, K. V. Shenoy, and M. Sahani.
% Dynamical segmentation of single trials from population neural data.
% In J. Shawe-Taylor, R. S. Zemel, P. Bartlett, F. C. N. Pereira, and
% K. Q. Weinberger, eds., Advances in Neural Information Processing
% Systems, vol. 24, pp. 756–764. Curran Associates, Inc., Red Hook, New
% York, 2011.\\
% Cited 64 times.


\item J. H. Macke, L. Büsing, J. P. Cunningham, B. M. Yu, K. V. Shenoy, and M. Sahani.
Empirical models of spiking in neural populations.
In J. Shawe-Taylor, R. S. Zemel, P. Bartlett, F. C. N. Pereira, and
K. Q. Weinberger, eds., Advances in Neural Information Processing
Systems, vol. 24, pp. 1350–1358. Curran Associates, Inc., Red Hook,
New York, 2011.\\
Introduced effective methods to identify dynamical systems underlying
population data.  Cited 210 times.
\end{itemize}
\noindent This remains an active thread of my research, with key recent
papers including:
\begin{itemize}[topsep=0pt,itemsep=0pt]
% \item L. Buesing, J. H. Macke, and M. Sahani.
% Spectral learning of linear dynamics from generalised-linear observations with application to neural population data.
% In P. Bartlett, F. C. N. Pereira, L. Bottou, C. J. C. Burges, and K. Q. Weinberger, eds., Advances in Neural Information Processing Systems, vol. 25, 2012.
% 
% \item J. H. Macke, L. Buesing, and M. Sahani.
% Estimating state and model parameters in state-space models of spike trains.
% In Z. Chen, ed., Advanced State Space Methods for Neural and Clinical Data. Cambridge University Press, 2015. 

\item L. Duncker, G. Bohner, J. Boussard, and M. Sahani.
Learning interpretable continuous-time models of latent stochastic dynamical systems.
In K. Chaudhuri and R. Salakhutdinov, eds., Proceedings of the 36th International Conference on Machine Learning, vol. 97 of Proceedings of Machine Learning Research, pp. 1726–1734. PMLR, Long Beach, California, USA, 09–15 Jun 2019. 
\item V. Rutten, A. Bernacchia, M. Sahani, and G. Hennequin.
Non-reversible gaussian processes for identifying latent dynamical structure in neural data.
In H. Larochelle, M. Ranzato, R. Hadsell, M. F. Balcan, and H.-T. Lin,
eds., Advances in Neural Information Processing Systems 33. Curran
Associates, Inc., 2020.
\item H. Soulat, S. Keshvarzi,  T. Margrie, and M. Sahani.
  Probabilistic Tensor Decomposition of Neural Population Spiking
  Activity.
  Advances in Neural Information Processing Systems 34. Curran
  Associates, Inc., 2021.
\end{itemize}
Many of the algorithms described in these papers form the basis of
capabilities to be incorporated into Bonsai as part of the proposed
project.


I have also developed a number of other neuroinformatics tools
relevant to the other capabiilties to be developed, and so my
expertise here will be valuable for evaluation and guidance.
%
Key examples include  decoding methods:
\begin{itemize}[topsep=0pt,itemsep=0pt]
  \item B. M. Yu, G. Santhanam, M. Sahani, and K. V. Shenoy.
Neural decoding for motor and communication prostheses.
In K. G. Oweiss, ed., Statistical Signal Processing for Neuroscience,
pp. 219–263. Elsevier, 2010.
\item G. Santhanam, B. M. Yu, V. Gilja, S. I. Ryu, A. Afshar, M. Sahani, and K. V. Shenoy.
Factor-analysis methods for higher-performance neural prostheses.
Journal of Neurophysiology, 102:1315–1330, 2009. 
\item B. M. Yu, C. Kemere, G. Santhanam, A. Afshar, S. I. Ryu, T. H. Meng, M. Sahani, and K. V. Shenoy.
Mixture of trajectory models for neural decoding of goal-directed movements.
Journal of Neurophysiology, 97(5):3763–3780, 2007. 
\end{itemize}
Cell segmentation methods for optical imaging:
\begin{itemize}[topsep=0pt,itemsep=0pt]
    \item M. Pachitariu, A. Packer, N. Pettit, H. Dalgleish, M. Hausser, M. Sahani.
    Extracting regions of interest from biological images with convolutional sparse block coding
    Advances in Neural Information Processing Systems 26, 1745-1753,
    2013.
  \item G. Bohner and M. Sahani.
    Convolutional higher order matching pursuit.
    In 2016 IEEE 26th International Workshop on Machine Learning for
    Signal Processing (MLSP), 2016.
\end{itemize}





% 3) How have you contributed to research teams and the development of
% others? % Examples might include: project management, supervision, mentoring or line management 
% contributions critical to the success of a team or team members, or where you exerted 
% strategic leadership in shaping the direction of a team, organisation, company or institution.

% From QQR:
\subsection{Contributions to teams and development}

To date, I have supervised a total of 14 PhDs and 14 postdocs.

Of the students, 8 now hold faculty (or group leader) positions at
Cambridge, CMU, Columbia, Copenhagen, Janelia (2), Pittsburgh and
Shanghai; 2 work in corporate research groups (DeepMind and Advanced
Bionics); and 4 recent graduates are currently in postdoctoral
positions (Aalto, Cambridge, CMU, Stanford).

Of the postdoctoral fellows, 6 have faculty positions at British
Columbia, Edinburgh, Oldenburg (2), Queensland and Tuebingen; 2 hold
independent research positions (Livermore Labs, Radboud); and 4 work
in industry.

Since 2017, I have been Director of the Gatsby Computational
Neuroscience Unit, leading strategy in research and teaching.  I have
created two new roles to support and develop new strategy, and have
recruited two new members of faculty who have expanded our research
portfolio. As Director, I sit in the Executive Leadership Committee of
the Faculty of Life Sciences at UCL.
 

% 4) How have you contributed to the wider research community?
% 
% Examples might include: how you have contributed to wider collaborations and networks 
% across disciplines, institutions, and / or countries, commitments such as editing, reviewing 
% and committee work, positions of responsibility, and activities which have contributed to the 
% improvement of research integrity or culture, or examples where you have shown visionary 
% strategic leadership in influencing a research agenda.  

\subsection{Contributions to wider research community}

I have been an active contributor to the academic community, shaping
the development of programme content for many scientific meetings,
critically assessing the intellectual content of research proposals
and awards applications, and reviewing the research and strategic
direction of various world-class research institutions.

I am a member of the Society for Neuroscience and IEEE, and in 2021
became a Fellow of the European Laboratory for Learning and
Intelligent Systems (ELLIS).

I served on the Board of Directors of the Computational Neuroscience
Organization between 2003-2006.
%
I was a member of the programme committees for 
%
the Neural Information Processing Systems (NeurIPS) meeting
(2004,2006); and the Computational and Systems Neuroscience (COSYNE)
conference (2007).
%
I was the Workshops Chair for NeurIPS in 2008 and for the Computational
Neuroscience Meeting from 1999-2003.
%
In 2009 I was the Programme Chair for COSYNE, taking over as General
Chair in  2010.
%
I co-founded the Neural Coding, Computation and Dynamics meeting in
2007, and was in the organising committee in 2015 and 2017.

I have held editorial roles at various journals including, Current
Opinion in Neurobiology (Special issue on Big Data and Neuroscience),
Neural Computation, Neural Systems and Circuits, Network: Computation
in Neural Systems, and Faculty of 1000, "Machine learning: life
sciences" collection.

I have taken various advisory roles for international review and award
panels. I reviewed the Donders Institute for Brain, Cognition and
Behaviour (Netherlands), The Edmond and Lily Safra Center for Brain
Sciences, Hebrew University (Israel), and the Barcelona Summer School
on Advanced Modelling of Behaviour (Spain). I have advised on awards
such as the Swartz Prize for Theoretical and Computational
Neuroscience (SFN, US), the Bernstein Prize (BMBF, Germany), and
COSYNE.

Recently, I have acted as an advisor to a new established
cross-disciplinary effort at the Indian Institute of Science in
Bangalore; visiting the site a number of times and hosting a joint
workshop here at UCL.

I have sat on various grant award panels for the BMBF and DFG
(Germany), and the (US) NSF/NIH collaborative research in
computational neuroscience (CRCNS) programme. I have also acted as a
reviewer for proposals for the Bernstein prize, the Einstein
Foundation, EPSRC, MRC, Novo Nordisk Foundation (Denmark) and NWO
(Netherlands).

% 5) How have you contributed to broader society? 
% Examples might include: engagement across the public and / or private sectors or with the 
% wider public, past research which has contributed to policy development or public 
% understanding, and other impacts across research, policy, practice and business, and other 
% examples of and how you have ensured your research reaches and influences relevant 
% audiences. 



% 6) Additional information  
% Any additional relevant information you wish to include in support of your capability to deliver, 
% which may include further information about key qualifications and relevant positions, 
% secondments, volunteering, or other relevant experience such as time spent in different 
% sectors. There is no need to provide information about career breaks, part-time working etc, 
% however if there are any details you do wish panel members to consider in their assessment 
% of the proposal they may be included here.
% 
}
