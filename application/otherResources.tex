\subsection*{Other resources in the subject area}

The space of technologies serving experimental control and behaviour monitoring
is large, and is traditionally occupied either by domain-specific graphical
user interfaces for control and recording of specific devices and experiment
types (e.g.\ Open Ephys
GUI\footnote{https://open-ephys.org/gui/},
Miniscope DAQ
Software\footnote{https://github.com/Aharoni-Lab/Miniscope-DAQ-QT-Software})
or by real-time control frameworks for specifying task logic using state
machine or similar formalisms (e.g. NIMH ML\footnote{https://monkeylogic.nimh.nih.gov/},
pyControl\footnote{https://pycontrol.readthedocs.io/en/latest/},
Autopilot\footnote{https://docs.auto-pi-lot.com/en/latest/},
Sanworks\footnote{https://sanworks.io/index.php}).
These dedicated interfaces are typically very comfortable for experimenters
operating within the specific domain that the tool is designed for, but tend to
become unwieldy when there is a need to introduce a new device or variation of
a task which is outside the scope of the framework. The alternative is to use a
more general programming language like Python or MATLAB, with the disadvantage
of the code being harder to understand, maintain, and change. Programming
languages like LabVIEW straddle the middle ground and provide a high-level and
flexible visual interface for composing data acquisition and control systems
themselves. Unlike Bonsai, however, the graphical elements of LabVIEW are
heterogeneous and very fine grained, creating the need for long and complicated
logical structures to implement even a simple experimental control system. By
providing an extremely simple, yet flexible, visual syntax, Bonsai opens the
opportunity even for complete non-programmers to design and successfully
customize relatively complex experiments from the ground up. It is mostly this
capability which has made Bonsai so attractive as a standard tool in
experimental neuroscience.

