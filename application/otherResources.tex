\subsection*{Other resources in the subject area}

The large field of technologies serving experimental control and behaviour monitoring is traditionally occupied either by domain-specific graphical
user interfaces for control and recording of specific devices and experiment
types (e.g.\ Open Ephys
GUI\footnote{open-ephys.org/gui/},
Miniscope DAQ
Software\footnote{github.com/ Aharoni-Lab/Miniscope-DAQ-QT-Software})
or by real-time control frameworks for specifying task logic using state
machine or similar formalisms (e.g. NIMH ML\footnote{monkeylogic.nimh.nih.gov/},
pyControl\footnote{pycontrol. readthedocs.io/ en/latest/},
Autopilot\footnote{docs.auto-pi-lot.com/en/latest/},
Sanworks\footnote{sanworks.io/index.php}).\break
These dedicated interfaces are typically very comfortable for experimenters in the specific domain for which the tool is designed, but can become unwieldy with the introduction of a new device or task variation from outside their usual scope. Alternatively, one can use a more general programming language such as Python or MATLAB, with the disadvantage
of the code being harder to understand, maintain, and change. Programming
languages like LabVIEW straddle the middle ground and provide a high-level, flexible visual interface for composing data acquisition and control systems. Unlike Bonsai, however, the graphical elements of LabVIEW are
heterogeneous and very fine grained, thus requiring long and complex
logical structures to implement even a simple experimental control system. By providing an extremely simple, yet flexible visual syntax, Bonsai provides the
opportunity for even complete non-programmers to design and successfully
customise relatively complex experiments from the ground up. It is this
capability in particular which has made Bonsai such an attractive standard tool in
experimental neuroscience.

