\textcolor{red}{I'm making this up -- Goncalo please comment/correct.}
%
As an open source environment developed over almost 10 years, Bonsai 
has well-established user-community-based management structures.
%
The core system and the majority of extension packages are hosted on a
publicly accessible website at (\url{bonsai-rx.org}.
%
Package installation and updates are supported by standalone package
management software included within the core Bonsai distribution.
%
This ensures that bugfixes and enhancements are easily available but
under full user control (to ensure working experiments are not
disrupted by changes in the system).

Best-practice code review procedures are followed prior to including
package updates.  \textcolor{red}{Goncalo/Joaquin -- could you expand
  on this?}

\textcolor{red}{Testing}

The software engineers recruited to this project will work with our
collaborators at Neurogears Ltd, to strengthen and maintain these
practices.  Once fully familiar with the Bonsai core archtecture they
will participate in review of pull requests and other code
contributions.  



Long-term sustainability






relies on distributed peer-reviewed management.  The collaborator 


To complement the management structure document, proposals may outline
how the resource management and advisory structures will operate including
outlining any review procedures.
Additionally, opportunities for staff training and support to ensure their
continuous development should be identified to safeguard the successful
delivery of the resource.



open source

user testing

workshops, user panels

