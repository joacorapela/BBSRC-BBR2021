\section{Resource management}
%
As an open source project developed over almost 10 years, Bonsai 
has well-established management practices built around the user community.
%
The source code for the core system and the majority of extension packages
is hosted on GitHub at \url{github.com/bonsai-rx}.
%
The documentation, including learning materials and all relevant community
links are published in a publicly accessible website at \url{bonsai-rx.org}.
%
A community user forum is available as a public Google Group with an active
user base of more than 700 users (Figure~\ref{fig:bonsai}f) and several
new questions and answers exchanged every day.
%
Package installation and updates are supported by standalone package
management software included within the core Bonsai distribution and hosted
in a shared centralized and curated repository.
%
This ensures that bugfixes and enhancements are made easily available but
also importantly allows for reproducible deployment of user environments
to ensure working experiments are not disrupted by changes in the system.
Large-scale experimental projects such as the International Brain Laboratory
have leveraged this capability of Bonsai to standardize data acquisition and
control software across dozens of laboratories around the world.
%
Community technology previews are made available through the package manager,
providing early adopters the possibility of opting-in to the latest features
prior to making them generally available. This allows for the community
to help identify bugs and report issues with new features on GitHub, thereby
distributing bug reporting and triage and overall improving the quality of
stable releases.

The core compiler and IDE is extensively tested, and also takes advantage of
a strong type system to greatly reduce bugs at compile-time. A set of best
practices for leveraging this infrastructure for building new packages is
in place.

The software engineers recruited to this project will work with our
collaborators at NeuroGEARS Ltd, to strengthen and maintain these
practices.  Once fully familiar with the Bonsai core architecture they
will participate in review of pull requests and other code contributions.

% To complement the management structure document, proposals may outline
% how the resource management and advisory structures will operate including
% outlining any review procedures.

% Additionally, opportunities for staff training and support to ensure their
% continuous development should be identified to safeguard the successful
% delivery of the resource.

\section{Long-term sustainability planning}

Bonsai is a freely available open-source resource, already widely adopted within
the systems neuroscience and behavioural science communities (see Fig~\ref{fig:bonsai}).  
%
Current contributors include other large open projects (OpenEphys, DeepLabCut),
as well as companies developing and supplying new hardware devices  (Neurophotometrics).
%
This level of community uptake, engagement, and commitment provides a foundation for
long-term sustainability of the resource.

NeuroGEARS Ltd is a keystone of this community and a key collaborator in the current proposal.
As a company founded by the original developer of Bonsai, it helps to support and expand the
user base well beyond the life of the current proposal. NeuroGEARS currently provides
quality control and curation for the core Bonsai code, as well as providing training and support
services for research institutions around the world who want to start using Bonsai.
Finally, NeuroGEARS also facilitates knowledge transfer across users and institutions adopting
Bonsai, by moderating the community user forums and collaborating with organisations
who want to build new innovative research projects on top of Bonsai, as is the case with the
current proposal.

The specific contributions developed as part of the current project will form new core
components and packages within the Bonsai ecosystem. The new language integration packages are
of particular relevance to the long-term sustainability of Bonsai, as they will dramatically
increase the interoperability of Bonsai with existing software development communities which are
already well established in basic science (MATLAB, Python, R).
%
The opportunity to develop these new packages with input from the core advisory panel of Bonsai,
and early public distribution and user testing within the SWC and planned workshops, will ensure
solid and broad user engagement from the beginning of the development cycle and will therefore
build momentum for their use, support and dissemination.
%
The close collaboration with the core maintainers of the Bonsai ecosystem at NeuroGEARS will
allow them to take over responsibility for maintenance and support of these packages beyond the
end of the project, following the same principles of open-source and free sharing of knowledge
that characterises the entire Bonsai ecosystem.

