\section{Resource management}
\textcolor{red}{I'm making this up -- Goncalo please comment/correct.}
%
As an open source environment developed over almost 10 years, Bonsai 
has well-established user-community-based management structures.
%
The core system and the majority of extension packages are hosted on a
publicly accessible website at \url{bonsai-rx.org}.
%
Package installation and updates are supported by standalone package
management software included within the core Bonsai distribution.
%
This ensures that bugfixes and enhancements are easily available but
under full user control (to ensure working experiments are not
disrupted by changes in the system).

Best-practice code review procedures are followed prior to including
package updates.  \textcolor{red}{Goncalo/Joaquin -- could you expand
  on this?}

\textcolor{red}{Testing}

The software engineers recruited to this project will work with our
collaborators at Neurogears Ltd, to strengthen and maintain these
practices.  Once fully familiar with the Bonsai core architecture they
will participate in review of pull requests and other code
contributions.  

% To complement the management structure document, proposals may outline
% how the resource management and advisory structures will operate including
% outlining any review procedures.

% Additionally, opportunities for staff training and support to ensure their
% continuous development should be identified to safeguard the successful
% delivery of the resource.

\section{Long-term sustainability planning}

Bonsai is a freely available open-source resource, already widely adopted within the systems neuroscience and behavioural science communities (see Fig~\ref{fig:bonsai}).  
%
Current contributors include other large open projects (OpenEphys, \textcolor{red}{\dots}), as well as companies developing and supplying new hardware devices  (\textcolor{red}{\dots}).
%
This level of community uptake, engagement, and commitment provides a foundation for long-term sustainability of the resource.

Neurogears Ltd is a keystone of this community and a key a collaborator in the current proposal.  As a company founded by the original developer of Bonsai, it helps to support and expand the user base well beyond the life of the current proposal.  Neurogears provides quality control and curation for core Bonsai code.  \textcolor{red}{other ways NG contributes ...} 

The specific contributions developed as part of the current project will form new components within the core Bonsai package. 
%
They will be developed with input from the advisory panel, early public distribution, user testing within the SWC, and planned workshops, ensuring close user engagement from the beginning of the development cycle and so building momentum for their use, support and dissemination.
%
\textcolor{red}{Can we say that NG will take over responsibility for maintenance and support beyond the project?}





