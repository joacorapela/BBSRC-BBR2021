In the past four years we have developed bonsai-rx.org as a hub for documentation and learning
resources (e.g. courses, video tutorials and examples) on how to use Bonsai. We have also engaged
with leading, relevant training efforts such as the CAJAL Neuroscience
Training Programme (Bonsai 0121) and the Transylvania Experimental Neuroscience
Summer School, to increase awareness of Bonsai and
its packages throughout the neuroscience community. With participation
at smaller invited venues, there have been on average 5 Bonsai courses
each year, with at least 20 students each; i.e.  around 100 neuroscience students annually are introduced to experimental tools directly through a
course in the Bonsai programming language.

Bonsai's 10th anniversary will take place in 2022. The proposed ML package will leverage planned events, and infrastructure to disseminate
awareness throughout the community, including presentations at conferences,
workshops, and training sessions, as well as electronic newsletters, forums and
social media.

The proposed development will facilitate interfacing with other programming languages
such as Python, R, and MATLAB. This will attract a whole new
community of Bonsai users with expertise in software development. To provide these new users with Bonsai's full power,
we plan to organise hackathons to accelerate integration of their existing data
analysis and ML algorithms and methods into Bonsai using the new
ML package.

 Given the modular and integrative nature of the ML package, we
 plan to advertise its development early in the project, to allow interested
 partners to contribute to key
 components of the platform, such as the Python and MATLAB integrations. We
 also want to promote co-development of algorithms by early adopters, in true collaborative fashion. Given its modular nature and built-in
 package manager capable of handling dependency resolution and curation of
 community contributed content, Bonsai is well positioned for such an approach.
