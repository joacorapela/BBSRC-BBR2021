
\documentclass[12pt]{letter}

\usepackage[margin=1.25in]{geometry}

\begin{document}

\begin{letter}{
    BBSRC-BBR Evaluation Committee\\
    Polaris House\\
    North Star Avenue\\
    Swindon, Wiltshire\\
    United Kingdom SN2 1UH
}
\opening{Dear Committee Members,}

We thank the reviewer for his/her comments, that we address below, indexed by
the corresponding sections.

\noindent\textbf{Excellence}

\texttt{The problem is that building high-level ML objects may hide caveats of
such algorithms and may make people who use them not understand what they are
actually doing.}

We share these concerns with the reviewer. Some data-analysis open-source
software packages allow users to fit complex statistical models by just
clicking a few buttons in a graphical user interface. And sometimes, this ease
of use leds users to apply these statistical models without sufficient
understanding. We want to avoid this problem in Bonsai.

We will invest substantial resources to educate Bonsai users about the
functionality of the machine learning methods added to Bonsai.  We will provide
extensive written and video documentation explaining these methods.  In
addition, in collaboration with methods authors, we will deliver talks in
workshops explaining the function of added methods.
%
The Gatsby Computational Neuroscience Unit is a centre of excellence in
statistical neuroscience and machine learning. Its contributions will be
essential to build simple and high-quality statistical documentation related to
the interfaced methods.

In addition, for every machine-learning method that we will integrate into
Bonsai, we will provide recommendations on how to report analysis results.
This documentation will be targeted to methods' users, as well as journal and
grant reviewers. It will inform these reviewers what they could expect from
statistical analysis perfromed with methods integrated into the Bonsai
ecosysstem.

Critical to help Bonsai users understand the function of the integrated machine
learning methods, and interpret their estimated models' parameters, is to
provide extensive visualisations of these parameters. We will provide such
visualisations.

The Bonsai interfaces that we will develop to machine learning methods will
allow Bonsai users to set any parameter of these methods, and the proposed
documentation will describe in detail the function of all parameters. Although,
we will provide good default values, users will be able to set any parameter
value. In this sense, Bonsai will not be a high-level interface to these
methods.

A central goal of Bonsai is to allow people without programming experience to
perform sophisticated experiments and, with the addition of the proposed
machine-learning functionality, to analyse their results. However, this does
not mean that non-programmers should be ignorant about how the statistical
methods that they are using work. We will help Bonsai users understand the
statistical methods that they use. Programming experience, or deep statistical
knowledge, should not required for this.

Our vision is that Bonsai will not only allow its users to run advanced
machine-learning methods, but it will also serve as an educational tool, that
will provide users deep understanding of what these methods do.

\noindent\textbf{Uniqueness and Availability}

\texttt{The added high-level ML tools are already available as free packages,
but this proposal puts them all together and within the Bonsai framework, so
the uniqueness is somewhat lower.}

It is true that the machine learning functionality that we propose to
incorporate into Bonsai is already available as free packages. However, people
without much programming experience will most commonly not be able to use this
functionality to control or analyse their experiments. Thus, a first unique
contribution from our proposal is that we will provide machine learning
functionality to people without programming experience, which will most
probably accelerate discovery.

Experimental control and neural data analysis are currently highly
disassociated. Most neuroscientists collect their data and only afterwards
analyse it. This split has had negative consequences for the progress of
neuroscience, as the vast majority of neural data analysis methods are offline
and do not provide key functionality for online experimental control. Hence, a
second unique contribution from our proposal is that it will generate the need
of online data analysis methods, which may generate advances in the field.

\end{letter}

\end{document}

