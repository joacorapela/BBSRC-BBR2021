
\documentclass[11pt]{letter}

\usepackage[hypertexnames=false,colorlinks=true,breaklinks]{hyperref}
\usepackage[a4paper, margin=2cm]{geometry}

\begin{document}

\begin{letter}{
}

Thank you for the reviewers' feedback and the opportunity to respond to
comments.

Response to reviewer 175134370:

Regarding the specific comment on reaching new user communities: the proposed
expansion of the Bonsai ecosystem, and the hackathons, workshops etc. mentioned
in the proposal, will be disseminated via numerous methods and fora. Many of
these methods have been successfully used to promote Bonsai thus far, including
professional networks, social media, and engagement with open-source hardware
projects such as Open-Ephys and Neurophotometrics, both of which have actively
promoted Bonsai for use with their systems. As part of the proposed extension,
we will demonstrate Bonsai and its new machine learning functionality at both
experimental and computational neuroscience venues (e.g., BNA's Festival of
Neuroscience, SFN's annual meeting, FENS meetings, Neuromatch Academy,
NeuroIPS, CoSyNe), to engage experimental neuroscience users, as well as new
neural data analysis method developers.

From SWC and GCNU, we will put together a comprehensive communications plan to
engage new communities with Bonsai. We will use a range of communications
channels including owned (SWC and GCNU websites, newsletters), shared (organic
social media) and paid (targeted/boosted social media and paid search)
channels. This could include publishing and publicising longer-form content on
how Bonsai is currently used by the global neuroscience community (e.g. at the
Allen Institute, Janelia Research Campus, etc.) and how the proposed expansion
will enhance this.

We will also empower current users to act as advocates for Bonsai, thereby
maximising reach through existing networks. Group Leaders will promote Bonsai
at conferences during research talks, and we will make targeted approaches to
Group Leaders not currently using the tools who may both benefit from and be
valuable promoters of Bonsai. This will include laboratories both within
neuroscience and from other fields— potentially leveraging user contacts in the
biological and drug screening groups already utilising Bonsai.

\pagebreak

Response to reviewer 155296694:

We share the concerns of the reviewer regarding people using machine learning
methods without understanding them. Some current data-analysis software
packages allow users to fit complex statistical models by just clicking a few
buttons in a graphical user interface. And sometimes, this ease of use leads
users to apply these statistical models without sufficient understanding. We
want to avoid this problem in Bonsai.

We will invest substantial resources to educate Bonsai users about the
functionality of the machine learning methods added to Bonsai. Furthermore, we will provide
extensive written and video documentation explaining the integrated methods. We are currently developing a technical reference  website for Bonsai which is open-source, integrated into the language editor, and open for community contributions. We aim to make extensive use of this website to raise awareness of the details of each method very close to the users.  In
addition, and in collaboration with method authors, we will deliver talks in
workshops explaining the function of added methods.
%
The Gatsby Computational Neuroscience Unit is a centre of excellence in
statistical neuroscience and machine learning. Its contributions will be
essential to building simple and accurate statistical documentation related to
the interfaced methods.

In addition, for every machine-learning method that we will integrate into
Bonsai, we will provide recommendations on how to report analysis results.
This documentation will be targeted to methods' users, as well as journal and
grant reviewers. It will inform these reviewers what they could expect from
statistical analysis performed with methods integrated into the Bonsai
ecosystem.

Critical to helping Bonsai users understand the function of the integrated machine
learning methods, and interpret their estimated models' parameters, is to
provide extensive visualisations of these parameters. We will provide such
visualisations by taking advantage of Bonsai's built-in data visualisation functionality.

The Bonsai interfaces that we will develop with machine learning methods will
allow Bonsai users to set any parameter of these methods, and the proposed
documentation will describe in detail the function of all parameters. Although,
we will provide good default values, users will be able to set any parameter
value. In this sense, Bonsai will not by itself be a high-level interface to these
methods.

A central goal of Bonsai is to allow people without programming experience to
perform sophisticated experiments and, with the addition of the proposed
machine-learning functionality, to analyse their results. However, this does
not mean that non-programmers should be ignorant about how the statistical
methods that they are using work. We will help Bonsai users understand the
statistical methods that they use. Programming experience, or deep statistical
knowledge, should not be required for this.

Our vision is that Bonsai will not only allow its users to run advanced
machine-learning methods, but it will also serve as an educational tool, that
will provide users deep understanding of machine learning functionality for
experimental control and neural data analysis.

Regarding the comment about the lower uniquess of our proposal by the fact that
the machine learning functionality that we proposed to add to Bonsai already
exists as free software package, we would like the make the following
clarifications.

It is not the aim of the proposed resource to develop new machine learning
functionality. We intend to build a resource that can easily integrate into the
experimental control loop existing reliable machine learning functionality
developed by the machine learing community. In this way, the
proposed resource should continuously provide to experimental neuroscientists
up-to-date machine learning functionality. In addition, the resource should
provide methods developers rich datasets on which to assess the performance of
their contributed methods. We believe that these contributions are unique.

Bonsai has a good track record of providing advanced functionality to a broad
community of users, for example in the existing integration with
DeepLabCut\footnote{\url{https://elifesciences.org/articles/61909}} or the
support for dynamic presentation of visual stimuli in
BonVision\footnote{\url{https://elifesciences.org/articles/65541}}. The
proposed machine learning extensions should significantly enhance these
contributions.

Experimental control and neural data analysis are currently highly
disassociated. Most neuroscientists collect their data and only afterwards
analyse it. This split has had negative consequences for the progress of
neuroscience, as the vast majority of neural data analysis methods are offline
and do not provide key functionality for online experimental control. Hence, a
second unique contribution from our proposal is that it will generate the need
of online data analysis methods, which may generate advances in the field. For
example, we have discussed the possibility of enabling and standardising
systems for online closed-loop control of behavioural tasks where the results
of statistical inference directly feedback to modulate the environment to which
animals are exposed. Such novel tasks remain very hard to implement at the
moment, and their reproducibility is complicated by the fact that there is no
standardised way to easily share such interactive systems. Bonsai is well
positioned to address this problem and we expect that the impact of such
practices will be very high in the community of experimental neuroscientists.

\pagebreak

A response to reviewer 206073742 is not required. 

\end{letter}

\end{document}

