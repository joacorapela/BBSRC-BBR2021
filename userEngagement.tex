In the past four years we have developed the bonsai-rx.org website with
documentation on how to use the language, and as a central hub for learning
resources, such as courses, video tutorials and examples. We have also engaged
with leading training efforts in the field such as the CAJAL Neuroscience
Training Programme (Bonsai 0121) and the Transylvania Experimental Neuroscience
Summer School (TENSS)  to increase awareness of the availability of Bonsai and
its packages throughout the neuroscience community. Together with participation
at smaller invited venues, there has been an average of 5 courses in Bonsai
each year, with at least 20 students each, so about 100 neuroscience students
every year get their introduction to experimental tools directly through a
course in the Bonsai programming language.

The development of this new machine learning package will be deeply integrated
into the Bonsai ecosystem which will complete its 10th anniversary in 2022, and
will thus leverage planned venues, events, and infrastructure to disseminate
awareness throughout the community, including presentations at conferences,
workshops, and training sessions, as well as electronic newsletter, forums and
social media articles.

The new infrastructure that we will build on Bonsai to support the machine
learning package will facilitate interfacing with other programming languages
such as Python, R, and MATLAB, which will therefore attract a whole new
community of users to Bonsai with expertise in software development. To take
advantage of this new community and provide to them the full power of Bonsai,
we plan to organize hackathons to accelerate integration of their existing data
analysis and machine learning algorithms and methods into Bonsai using the new
machine learning package.

 Given the modular and integrative nature of the machine learning package, we
 plan to advertise its development early in the project, so that interested
 partners have the opportunity to contribute to the requirements of key
 components of the platform, such as the Python and MATLAB integrations. We
 also want to promote the co-development of algorithms by early adopters of the
 package, in true collaborative fashion. Given its modular nature and built-in
 package manager capable of handling dependency resolution and curation of
 community contributed content, Bonsai is well positioned for such an approach.
